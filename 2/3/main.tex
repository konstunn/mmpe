
\documentclass[a4paper,14pt]{extarticle}

\usepackage{cmap}

\usepackage[T2A]{fontenc}
\usepackage[utf8x]{inputenc}
\usepackage[russian]{babel}

\usepackage[a4paper,margin=1.5cm,footskip=1cm,left=2cm,right=1.5cm,top=1.5cm
  ,bottom=2.0cm]{geometry}
\usepackage{textcase}
\usepackage{csquotes}
\usepackage{enumitem}

\usepackage[labelsep=period,justification=centering]{caption}

\usepackage{graphicx}
\graphicspath{ {figure/} }

\usepackage{amsmath}
\usepackage{pgfplots}

\usepackage{float}

\usepackage{indentfirst}

\usepackage{textgreek}

\usepackage{pythontex}

\usepackage{comment}

\usepackage{todonotes}

% 
\setlist[description]{leftmargin=\parindent,labelindent=\parindent}

\renewcommand{\baselinestretch}{1.5}

\usepackage[titletoc,title]{appendix}

\DeclareMathOperator{\Sp}{Sp}

% 
\renewcommand{\vec}[1]{#1}

\newcommand{\pd}[2]{\frac{\partial #1}{\partial #2}}
\newcommand{\pdpk}[1]{\pd{#1}{\theta_i}}

\newcommand{\inv}[1]{#1^{-1}}

\newcommand{\eps}{\varepsilon}

\begin{document}

\setcounter{secnumdepth}{0}

\begin{titlepage}

  \begin{center}
    Новосибирский государственный технический университет
    
    Факультет прикладной математики и информатики
    
    Кафедра теоретической и прикладной информатики
    
    \vspace{250pt}
    
    \textbf{\Large{Лабораторная работа № 3}}
    \medbreak
		<<Вычисление информационной матрицы Фишера для модели стохастической
		линейной непрерывно-дискретной системы>>
		\medbreak 
		по дисциплине \\
    \medbreak
    <<Математические методы планирования эксперимента>>
    \vspace{100pt}
  \end{center}

  \begin{flushleft}
    \begin{tabbing}
      Группа:\qquad\qquad \= ПММ-61\\
      Студент:            \> Горбунов К. К.\\
      Преподаватель:      \> Чубич В. М.\\
    \end{tabbing}
  \end{flushleft}

  \begin{center}
    \vspace{\fill}
    Новосибирск, 2017 г.
  \end{center}

\end{titlepage}

\newpage

\tableofcontents

\newpage

\section{Цель работы}

Изучить алгоритм вычисления информационной матрицы Фишера (ИМФ) для моделей
стохастических линейных непрерывно-дискретных систем, отработать навыки
программирования на реализации соответствующей процедуры.

\section{Порядок выполнения лабораторной работы}

\begin{enumerate}

\item Изучить соответствующий теоретический материал.

\item Последовательно выполнить все задания к лабораторной работе.

\end{enumerate}

\section{Задание к лабораторной работе}

\begin{enumerate}

	\item Реализовать программную процедуру вычисления ИМФ для указанного в цели
		работы класса моделей.

	\item Проверить правильность реализации алгоритмов и работоспособность
		программ.

\end{enumerate}

\section{Теоретический материал}

\subsection{Описание модельной структуры}

Рассмотрим модель стохастической динамической линейной \newline
непрерывно-дискретной системы в пространстве состояний в виде \cite{mono}:
\begin{equation}
  \label{eq:initmod}
  \left\{ 
    \begin{array}{lll}
			\pd{\vec{x}(t)}{t} &= F \vec{x}(t) + C \vec{u}(t) + G \vec{w}(t),
				& t \in [t_0, T], \\
      \vec{y}(t_{k+1}) &= H \vec{x}(t_{k+1}) + \vec{v}(t_{k+1}), 
				& k = 0,\ldots, N-1
    \end{array} 
  \right. 
\end{equation}

Здесь:
\begin{description}
  \item [$\vec{x}(t)$] -- $n$-вектор состояния;
  \item [$F$] -- матрица перехода состояния;
  \item [$\vec{u}(t)$] -- $r$-вектор управления (входного воздействия);
  \item [$C$] -- матрица управления;
  \item [$\vec{w}(t)$] -- $p$-вектор возмущений;
  \item [$G$] -- матрица влияния возмущений;
  \item [$H$] -- матрица наблюдения;
  \item [$\vec{v}(t_{k+1})$] -- $m$-вектор шума измерений;
  \item [$\vec{y}(t_{k+1})$] -- $m$-вектор наблюдений (измерений) отклика;
\end{description}

$F, C, G, H$ --- матрицы соответствующих размеров.

\bigskip
Априорные предположения:
\begin{itemize}
\item $F$ устойчива;
\item пары $(F, C)$ и $(F, G)$ управляемы;
\item пара $(F, H)$ --- наблюдаема;
\item случайные процессы $\{\vec{w}(t), t \in [t_0, T]\}$ и
	$\{\vec{v}(t_{k}), k = 1, 2, \ldots, N\}$ являются стационарными белыми
		гауссовскими шумами, причем:
\[
	E[\vec{w}(t)] = 0,\ E[\vec{w}(t)\vec{w}^{T}(\tau)] = Q \delta(t-\tau)\,
\]
\[
	E[\vec{v}(t_k)] = 0,\ E[\vec{v}(t_k)\vec{v}^{T}(t_j)] = R \delta_{k,j}\,
\]
\[
	E[\vec{v}(t_k)\vec{w}^{T}(t)] = 0,\qquad \forall t_k, t,
\]
		где $\delta(t-\tau)$ --- дельта-функция Дирака, $\delta_{jk}$ --- символ
		Кронекера;

\item начальное состояние $\vec{x}(t_0)$ имеет нормальное распределение с
параметрами $\overline{\vec{x}}(t_0)$ и $P(t_0)$ и не коррелирует с
		$\vec{w(t)}$ и $\vec{v_k}$ при любых значениях $t$ и $k$.

Будем считать, что подлежащие оцениванию параметры $\theta = (\theta_1,
\theta_2, \ldots, \theta_s)$ могут входить в элементы матриц $F, C, G, H, Q, R,
P(t_0)$ и в вектор $\overline{\vec{x}}(t_0)$ в различных комбинациях.

\end{itemize}

\subsection{Информационная матрица Фишера (ИМФ)}

\newcommand{\pred}[0]{t_{k+1}|t_{k}}
\newcommand{\est}[0]{t_k|t_k}
\newcommand{\fut}[0]{t_{k+1}}
\newcommand{\upd}[0]{t_{k+1}|t_{k+1}}
\newcommand{\ol}[1]{\overline{#1}}
\newcommand{\Th}[0]{\Theta}

\newcommand{\sumlim}[2]{\sum\limits_{#1}^{#2}}

Для математической модели (\ref{eq:initmod}) и соответствующими априорными
предположениями, описанными выше, элементы информационной матрицы одноточечного
плана определяются выражением
% TODO: 
\begin{multline}
	\label{eq:fim}
	\hspace{0.5\textwidth} M_{ij}(U;\Th) =\\
		= \sumlim{k=0}{N-1} \left\{ \Sp \left[ 
		S_0 \left(\Sigma_A(\fut) + \ol{x}_A(\fut) \ol{x}^T_A(\fut) \right)
		S_0^T \pd{H^T}{\theta_j} \inv{B}(t_k) \pd{H}{\theta_i} \right] + \right. \\
		\left.
		+ \Sp\left[ S_0 \left( \Sigma_A(\fut) + \ol{x}_A(\fut)
		\ol{x}_A^T(\fut) \right) S_j^T H^T \inv{B}(t_k) \pd{H}{\theta_i}
		\right] + \right. \\ \left.
		+ \Sp\left[ S_i \left( \Sigma_A(\fut) + \ol{x}_A(\fut) \ol{x}^T_A(\fut)
		\right) S_0^T \pd{H^T}{\theta_j} \inv{B}(t_k) H \right] + \right. \\ \left.
		+ \Sp \left[ S_i \left( \Sigma_A(\fut) + \ol{x}_A(\fut) \ol{x}^T_A(\fut)
		\right) S_j^T H^T \inv{B}(t_k) H \right] + \right. \\ \left.
		+ \Sp \left[ \pd{B(t_k)}{\theta_i} \inv{B}(t_k) \pd{B(t_k)}{\theta_j} 
		\right]  \right\},\qquad i,j = 1,2,\ldots,s.
\end{multline}
\newcommand\undermat[2]{%
	  \makebox[0pt][l]{$\smash{\underbrace{\phantom{%
			    \begin{matrix}#2\end{matrix}}}_{\text{$#1$}}}$}#2}
\[
	S_i = {\left[
	\begin{array}{rrr|r|rrr}
		\undermat{i}{O & \cdots & O} & I & 
			\undermat{s-i}{O & \cdots & O} \\
	\end{array}
	\right]}_{n \times n(s+1)},\ i = 0, 1, \ldots, s.
\]

\subsubsection{Алгоритм вычисления ИМФ}
\begin{enumerate}
	\item Положим $k = 0,\ M(\Theta) = 0$. \\
		Для заданного $\Theta = (\theta_1, \theta_2, \ldots, \theta_s)$ вычислим:
\begin{align*}
	F,\ \pd{F}{\theta_i},\ i=1,2,\ldots,s; \qquad &
	Q,\ \pd{Q}{\theta_i},\ i=1,2,\ldots,s; \\
	C,\ \pd{C}{\theta_i},\ i=1,2,\ldots,s; \qquad &
	R,\ \pd{R}{\theta_i},\ i=1,2,\ldots,s; \\
	G,\ \pd{G}{\theta_i},\ i=1,2,\ldots,s; \qquad &
	\ol{x}_0,\ \pd{\ol{x}_0}{\theta_i},\ i=1,2,\ldots,s; \\
	H,\ \pd{H}{\theta_i},\ i=1,2,\ldots,s; \qquad &
	P(t_k|t_k) = P_0,\ 
		\pd{P(t_k|t_k)}{\theta_i} = \pd{P_0}{\theta_i},\ i=1,2,\ldots,s.
\end{align*}

\newpage

\newcommand{\deriv}[2]{\frac{d #1}{d #2}}

\item Решив следующие матричные дифференциальные уравнения (ДУ) в обратном
	времени:
	\begin{align*}
		\deriv{\Phi(t_{k+1}, \tau)}{\tau} &= - \Phi(t_{k+1}, \tau) F(\tau), \tau \in
			[t_k, t_{k+1}]; \\
		\Phi(t_{k+1}, t_{k+1}) &= I, 
	\end{align*}
	\begin{align*}
		\deriv{}{\tau} \left( \pd{\Phi(t_{k+1}, \tau)}{\theta_i} \right) &=
			- \pd{\Phi(t_{k+1}, \tau)}{\theta_i} F(\tau) - \Phi(t_{k+1}, \tau)
			\pd{F(\tau)}{\theta_i}, \tau \in [t_k, t_{k+1}]; \\
		\pd{\Phi(t_{k+1}, t_{k+1})}{\theta_i} &= O,
	\end{align*}
	найти \{$\Phi(t_{k+1}, \tau), \tau \in [t_k, t_{k+1}]\}$ и 
		$\left\{ \pd{\Phi(t_{k+1}, \tau)}{\theta_i},\ i=1,2,\ldots,s,\ \tau \in
		[t_k, t_{k+1}] \right\}$.

	\item Если $k=0$, вычислить $\ol{x}_A(t_{k+1})$ и $\Sigma_A(t_{k+1})$ по
		следующим формулам:
		\begin{align*}
			\ol{x}_A(t_{k+1}) = \hspace{0.4\textwidth} \\
		\begin{bmatrix}
			\Phi(t_1, t_0) \ol{x}(t_0) + \int\limits_{t_0}^{t_1} \Phi(t_1, \tau)
				C u(\tau) d\tau \\
			\pd{\Phi(t_1, t_0)}{\theta_1} \ol{x}(t_0) + \Phi(t_1, t_0)
				\pd{\ol{x}(t_0)}{\theta_1} + \int\limits_{t_0}^{t_1} 
				\left[ \pd{\Phi(t_1, \tau)}{\theta_1} Cu(\tau) + \Phi(t_1, \tau) 
				\pd{C}{\theta_1} u(\tau) \right] d\tau \\
			\vdots \\
			\pd{\Phi(t_1, t_0)}{\theta_s} \ol{x}(t_0) + \Phi(t_1, t_0)
				\pd{\ol{x}(t_0)}{\theta_s} + \int\limits_{t_0}^{t_1} 
				\left[ \pd{\Phi(t_1, \tau)}{\theta_s} Cu(\tau) + \Phi(t_1, \tau) 
				\pd{C}{\theta_s} u(\tau) \right] d\tau 
		\end{bmatrix},
	\end{align*}
		\[
			\Sigma_A(t_{k+1}) = O,
		\]
	и перейти на шаг 7.

	\item Найти $\tilde{K}(t_{k+1}, t_k)$ при помощи выражения 
	\[
		\tilde{K}(t_{k+1}, t_k) = \Phi(t_{k+1}, t_k) K(t_k),
	\]
	и $\left\{ \pd{\tilde{K}(t_{k+1}, t_k)}{\theta_i},\ i=1,2,\ldots,s \right\}$
	по формуле
		\[
			\pd{\tilde{K}(t_{k+1}, t_k)}{\theta_i} = 
				\pd{\Phi(t_{k+1}, t_k)}{\theta_i} K(t_k) + \Phi(t_{k+1}, t_k)
				\pd{K(t_k)}{\theta_i}.
		\]
	\item Сформировать матрицы $\Phi_A(t_{k+1}, t_k),\ K_A(t_{k+1}, t_k)$ и
		вектор $a_A(t_{k+1}, t_k)$, воспользовавшись следующими равенствами:
		\[
			\Phi_A(t_{k+1}, t_k) =
			\begin{bmatrix}
				\Phi(t_{k+1}, t_k) & O & \ldots & O \\
				\pd{\Phi(t_{k+1}, t_k)}{\theta_1} - \tilde{K}(t_{k+1}, t_k)
					\pd{H}{\theta_1} & \phi_A(t_{k+1}, t_k) & \ldots & O \\
				\vdots & \vdots & \ddots & \vdots \\
				\pd{\Phi(t_{k+1}, t_k)}{\theta_s} - \tilde{K}(t_{k+1}, t_k)
					\pd{H}{\theta_s} & O & \ldots & \phi_A(t_{k+1}, t_k)
			\end{bmatrix},
		\]
		\[
			\phi_A(t_{k+1}, t_k) = \Phi(t_{k+1}, t_k) - \tilde{K}(t_{k+1}, t_k) H,
		\]
		\[
			K_A(t_{k+1}, t_k) =
			\begin{bmatrix}
				\tilde{K}(t_{k+1}, t_k) \\
				\pd{\tilde{K}(t_{k+1}, t_k)}{\theta_1} \\
				\vdots \\
				\pd{\tilde{K}(t_{k+1}, t_k)}{\theta_s} \\
			\end{bmatrix},
		\]
		\[
			a_A(t_{k+1}, t_k) =
			\begin{bmatrix}
				\int\limits_{t_k}^{t_{k+1}} \Phi(t_{k+1}, \tau) a(\tau) d\tau \\
				\int\limits_{t_k}^{t_{k+1}} \left[ \pd{\Phi(t_{k+1}, \tau)}{\theta_1} 
					a(\tau) + \Phi(t_{k+1}, \tau) \pd{a(\tau)}{\theta_1} \right] d\tau \\
				\vdots \\
				\int\limits_{t_k}^{t_{k+1}} \left[ \pd{\Phi(t_{k+1}, \tau)}{\theta_s} 
					a(\tau) + \Phi(t_{k+1}, \tau) \pd{a(\tau)}{\theta_s} \right] d\tau 
			\end{bmatrix},
		\]
		\[
			a(t) = C u(t).
		\]

	\item Вычислить $\ol{x}_A(t_{k+1})$ и $\Sigma_A(t_{k+1})$ по следующим
		формулам:
		\[
			\ol{x}_A(t_{k+1}) = \Phi(t_{k+1}, t_k) \ol{x}_A(t_k) + a_A(t_{k+1}, t_k),
		\]
		\[
			\Sigma_A(t_{k+1}) = \Phi_A(t_{k+1}, t_k) \Sigma_A(t_k) \Phi_A^T(t_{k+1},
			t_k) + K_A(t_{k+1}, t_k) B(t_k) K_A^T(t_{k+1}, t_k).
		\]

	\newpage

	\item Найти $P(t_{k+1}| t_k)$ и $\left\{\pd{P(t_{k+1}| t_k)}{\theta_i},\
		i=1,2,\ldots,s;\right\}$ путем решения матричного ДУ
		\begin{gather*}
	\frac{d}{d\tau}	
	\begin{bmatrix}
		P(\tau|t_k) \\
		\pd{P(\tau|t_k)}{\theta_1} \\
		\vdots \\
		\pd{P(\tau|t_k)}{\theta_s} \\
	\end{bmatrix} = \\
	\begin{bmatrix}
		F P(\tau|t_k) + P(\tau|t_k) F^T + GQG^T \\
		\pd{F}{\theta_1} P(\tau|t_k) + F \pd{P(\tau|t_k)}{\theta_1} +
			\pd{P(\tau|t_k)}{\theta_1} F^T + P(\tau|t_k) \pd{F^T}{\theta_1} + \\
			+ \pd{G}{\theta_1} Q G^T + G \pd{Q}{\theta_1} G^T +
			G Q \pd{G^T}{\theta_1} \\
		\vdots \\
		\pd{F}{\theta_s} P(\tau|t_k) + F \pd{P(\tau|t_k)}{\theta_s} +
			\pd{P(\tau|t_k)}{\theta_s} F^T + P(\tau|t_k) \pd{F^T}{\theta_s} + \\
			+ \pd{G}{\theta_s} Q G^T + G \pd{Q}{\theta_s} G^T +
			G Q \pd{G^T}{\theta_s}  
	\end{bmatrix} = \\ =
	\begin{bmatrix}
		F & O & \cdots & O \\
		\pd{F}{\theta_1} & F & \cdots & O \\
		\vdots & \vdots & \ddots & \vdots \\
		\pd{F}{\theta_s} & O & \cdots & F
	\end{bmatrix}
	\begin{bmatrix}
		P(\tau|t_k) \\
		\pd{P(\tau|t_k)}{\theta_1} \\
		\vdots \\
		\pd{P(\tau|t_k)}{\theta_s}
	\end{bmatrix} + \\ +
	\begin{bmatrix}
		P(\tau|t_k) & O & \ldots & O \\
		\pd{P(\tau|t_k)}{\theta_1} & P(\tau|t_k) & \ldots & O \\
		\vdots & \vdots & \ddots & \vdots \\
		\pd{P(\tau|t_k)}{\theta_s} & O & \ldots & P(\tau|t_k)
	\end{bmatrix}
	\begin{bmatrix}
		F^T \\
		\pd{F^T}{\theta_1} \\
		\vdots \\
		\pd{F^T}{\theta_s}
	\end{bmatrix} + \\
	\begin{bmatrix}
		G Q G^T \\
		\pd{G}{\theta_1} Q G^T \\
		\vdots \\
		\pd{G}{\theta_s} Q G^T 
	\end{bmatrix} +
	\begin{bmatrix}
		O \\
		G \pd{Q}{\theta_1} G^T \\
		\vdots \\
		G \pd{Q}{\theta_s} G^T
	\end{bmatrix} +
	\begin{bmatrix}
		O \\
		G Q \pd{G^T}{\theta_1} \\
		\vdots \\
		G Q \pd{G^T}{\theta_s}
	\end{bmatrix},\ \tau \in [t_k, t_{k+1}],
\end{gather*}
при известных значениях $P(t_k|t_k)$ и
	$\left\{ \pd{P(t_k| t_k)}{\theta_i},\ i = 1,2,\ldots,s \right\}$.

\item Вычислить
	\[
		B(t_{k+1}) = H P(t_{k+1}|t_k) H^T + R,
	\]
	\begin{align*}
		\pd{B(t_{k+1})}{\theta_i} = \pd{H}{\theta_i} P(t_{k+1}|t_k) H^T + 
			H \pd{P(t_{k+1}|t_k)}{\theta_i} H^T + H P(t_{k+1}|t_k)
			\pd{H^T}{\theta_i} + \pd{R}{\theta_i}, \\
		i = 1,2,\ldots,s;
	\end{align*}
	\[
		K(t_{k+1}) = P(t_{k+1}|t_k) H^T \inv{B(t_{k+1})}
	\]
	\begin{multline*}
		\pd{K(t_{k+1})}{\theta_i} = \pd{P(t_{k+1}|t_{k})}{\theta_i} H^T 
			\inv{B}(t_{k+1}) +
			P(t_{k+1}|t_{k}) \pd{H^T}{\theta_i} \inv{B}(t_{k+1}) + \\ 
			- P(t_{k+1}|t_{k}) H^T
			\inv{B}(t_{k+1}) \pd{B(t_{k+1})}{\theta_i} \inv{B}(t_{k+1}),\ i = 1, 2, \ldots, s.
	\end{multline*}
	\[
		P(t_{k+1}|t_{k+1}) = [ I - K(t_{k+1}) H ] P(t_{k+1}|t_k),
	\]
	\begin{multline*}
		\pd{P(t_{k+1}|t_{k+1})}{\theta_i} =
			[ I - K(t_{k+1}) H ] \pd{P(t_{k+1}|t_k)}{\theta_i} + \\ -
			\left[
				\pd{K(t_{k+1})}{\theta_i} H + K(t_{k+1}) \pd{H}{\theta_i}
			\right]
			P(t_{k+1}|t_k).
	\end{multline*}

\item Используя выражение (\ref{eq:fim}), получить приращение
	$\Delta M(\Theta)$, соответствующее текущему значению $k$.

\item Положить $M(\Theta) = M(\Theta) + \Delta M(\Theta)$.

\item Увеличить $k$ на единицу. Если $k \le N-1$, перейти на шаг 2. В противном 
	случае закончить процесс.

\end{enumerate}

% imports
\begin{pythontexcustomcode}{py}
import numpy as np
import dill as pkl
from model.model import Model
import pylatex
\end{pythontexcustomcode}

\section{Примеры}

% TODO: print numpy arrays here instead of hardcoded matrices
\[
  \begin{pmatrix} x_1(\fut) \\ x_2(\fut) \end{pmatrix} =
  \begin{pmatrix} \theta_1 & 1 \\ 0 & 0.5 \end{pmatrix}
  \begin{pmatrix} x_1(t_k) \\ x_2(t_k) \end{pmatrix}
  + 
	\begin{pmatrix} \theta_2 \\ 1 \end{pmatrix}
	u(t_k) 
  + \begin{pmatrix} w_1(t_k) \\ w_2(t_k) \end{pmatrix}
\]
\[
  y(\fut) =
	\begin{pmatrix} 1 & 0 \end{pmatrix}
	\begin{pmatrix} x_1(\fut) \\ x_2(\fut) \end{pmatrix} +
  \begin{pmatrix} v_1(\fut) \\ v_2(\fut) \end{pmatrix}
\]
\[
  \overline{x}(0) = \begin{pmatrix} 0 \\ 0 \end{pmatrix},\
  P(0) = \begin{pmatrix} 0.1 & 0 \\ 0 & 0.1 \end{pmatrix},\
  Q = \begin{pmatrix} 0.1 & 0 \\ 0 & 0.1 \end{pmatrix},\
  R = 0.1 
\]

Истинные значения параметров
$\theta_{true} = \begin{pmatrix} 0.5 & 0.5 \end{pmatrix}$ \\

\newpage
\subsubsection{Оценивание параметров по произвольному начальному плану}

Спектр непрерывного начального плана:

\begin{pycode}[][fontsize=\small]
filename = './data/plan0.pkl'
with open(filename, 'rb') as f:
    plan0 = pkl.load(f)

def print_plan_tex(plan0):
	print('$')
	print('U = \\left\{')

	for i in range(plan0[0].shape[0]-1):
		x = plan0[0][i]
		xx = pylatex.Matrix(np.round(x.T, 2), mtype='b').dumps()
		print(xx)
		print(',\ ')

	xx = pylatex.Matrix(np.round(plan0[0][-1].T), mtype='b').dumps()
	print(xx)
			
	print('\\right\}')
	print('$')

print_plan_tex(plan0)
\end{pycode}

Веса непрерывного начального плана:
\begin{pycode}
p = np.array(plan0[1], ndmin=2)
p = pylatex.Matrix(p, mtype='b').dumps()
print('$ p = ')
print(p)
print('$')
\end{pycode}

Длина точки плана $N = 20$.

Область планирования: $u(t_k) \in [ -1,\ 1 ],\ k = 0, 1, \ldots, N$.

Область оценивания: $ 0.25 \le \theta_i \le 0.75,\ i = 1, 2 $.

Общее число запусков $\nu = 10$. \\

Веса соответствующего дискретного начального плана:
\begin{pycode}
def round_w(p, v):
	import operator
	p = np.array(p)
	q = len(p)
	sigmaI = np.ceil((v - q) * p)  # 1
	sigmaII = np.floor(v * p)
	vI = v - np.sum(sigmaI)  # 2
	vII = v - np.sum(sigmaII)
	if vI < vII:
			sigma = sigmaI
			v1 = int(vI)
	else:
			sigma = sigmaII
			v1 = int(vII)

	s = np.zeros(q)

	vps = v * p - sigma
	vps_id = [i for i in range(len(vps))]
	vps_t = [(val, key) for val, key in zip(vps, vps_id)]

	vps_t = sorted(vps_t, key=operator.itemgetter(0))
	sorted_id = [elem[1] for elem in vps_t]

	for j in range(q):
			if vps_id[j] in sorted_id[:v1]:
					s[j] = 1
			else:
					s[j] = 0

	p = (sigma + s) / v
	return p

p = round_w(plan0[1], 10)
p = np.array(p, ndmin=2)
p = pylatex.Matrix(p, mtype='b').dumps()
print('$ p = ')
print(p)
print('$')
\end{pycode}

Полученные оценки параметров:
$\hat{\theta} = \begin{bmatrix} 0.342 & 0.559 \end{bmatrix}$.

\newcommand{\rtol}[2]{\frac{||\hat{#1}#2 -
{\hat{#1}#2_{true}}||}{||\hat{#1}#2_{true}||}}

Относительная погрешность в пространстве параметров:
\[
	\delta_{\theta} = \rtol{\theta}{} = 23.9\%.
\]

Средняя относительная погрешность в пространстве откликов:
\[
	\overline{\delta}_Y = \frac{1}{\nu}
		\sum\limits_{i=1}^{q}\sum\limits_{j=1}^{k_i} \rtol{Y}{^{i,j}} = 11.8 \%.
\]

\subsubsection{Синтез оптимального плана и оценивание параметров по плану}

По проведению двойственной процедуры планирования входных сигналов был получен
некоторый оптимальный план. В результате проведения идентификационных экспериментов по
данному плану были получены оценки параметров.

\newpage
Спектр оптимального плана:

\begin{pycode}
filename = './data/plan1.pkl'
with open(filename, 'rb') as f:
    plan1 = pkl.load(f)

print_plan_tex(plan1)
\end{pycode}

То есть в плане одна точка, соответственно, все запуски $\nu$ будут приходиться
на неё.

Полученные оценки параметров:
$\hat{\theta} = \begin{bmatrix} 0.46 & 0.47 \end{bmatrix}$.

Относительная погрешность в пространстве параметров:
\[
	\delta_{\theta} = \rtol{\theta}{} = 6.7\%.
\]

Средняя относительная погрешность в пространстве откликов:
\[
	\overline{\delta}_Y = \frac{1}{\nu}
		\sum\limits_{i=1}^{q}\sum\limits_{j=1}^{k_i} \rtol{Y}{^{i,j}} = 2.6 \%.
\]

\section{Выводы}

По результатам эксперимента видно уменьшение погрешности оценок при оценивании
по оптимальному плану. Таким образом подтверждается эффективность процедуры
активной параметрической идентификации.

\begin{thebibliography}{9}

\begin{sloppypar}

\bibitem{mono} Активная параметрическая идентификация стохастических линейных
  систем: монография / В.И. Денисов, В.М. Чубич, О.С. Черникова, Д.И. Бобылева.
    --- Новосибирск : Изд-во НГТУ, 2009. --- 192 с.
    (Серия <<Монографии НГТУ>>).

\end{sloppypar}

\end{thebibliography}

\end{document}

# vim: ts=2 sw=2
