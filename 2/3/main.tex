
\documentclass[a4paper,14pt]{extarticle}

\usepackage{cmap}

\usepackage[T2A]{fontenc}
\usepackage[utf8x]{inputenc}
\usepackage[russian]{babel}

\usepackage[a4paper,margin=1.5cm,footskip=1cm,left=2cm,right=1.5cm,top=1.5cm
  ,bottom=2.0cm]{geometry}
\usepackage{textcase}
\usepackage{csquotes}
\usepackage{enumitem}

\usepackage[labelsep=period,justification=centering]{caption}

\usepackage{graphicx}
\graphicspath{ {figure/} }

\usepackage{amsmath}
\usepackage{pgfplots}

\usepackage{float}

\usepackage{indentfirst}

\usepackage{textgreek}

\usepackage{pythontex}

\usepackage{comment}

\usepackage{todonotes}

% 
\setlist[description]{leftmargin=\parindent,labelindent=\parindent}

\renewcommand{\baselinestretch}{1.5}

\usepackage[titletoc,title]{appendix}

\DeclareMathOperator{\Sp}{Sp}

% 
\renewcommand{\vec}[1]{#1}

\newcommand{\pd}[2]{\frac{\partial #1}{\partial #2}}
\newcommand{\pdpk}[1]{\pd{#1}{\theta_i}}

\newcommand{\inv}[1]{#1^{-1}}

\newcommand{\eps}{\varepsilon}

\begin{document}

\setcounter{secnumdepth}{0}

\begin{titlepage}

  \begin{center}
    Новосибирский государственный технический университет
    
    Факультет прикладной математики и информатики
    
    Кафедра теоретической и прикладной информатики
    
    \vspace{250pt}
    
    \textbf{\Large{Лабораторная работа № 3}}
    \medbreak
		<<Вычисление информационной матрицы Фишера для модели стохастической
		линейной непрерывно-дискретной системы>>
		\medbreak 
		по дисциплине \\
    \medbreak
    <<Математические методы планирования эксперимента>>
    \vspace{100pt}
  \end{center}

  \begin{flushleft}
    \begin{tabbing}
      Группа:\qquad\qquad \= ПММ-61\\
      Студент:            \> Горбунов К. К.\\
      Преподаватель:      \> Чубич В. М.\\
    \end{tabbing}
  \end{flushleft}

  \begin{center}
    \vspace{\fill}
    Новосибирск, 2017 г.
  \end{center}

\end{titlepage}

\newpage

\tableofcontents

\newpage

\section{Цель работы}

Изучить алгоритм вычисления информационной матрицы Фишера (ИМФ) для моделей
стохастических линейных непрерывно-дискретных систем, отработать навыки
программирования на реализации соответствующей процедуры.

\section{Порядок выполнения лабораторной работы}

\begin{enumerate}

\item Изучить соответствующий теоретический материал.

\item Последовательно выполнить все задания к лабораторной работе.

\end{enumerate}

\section{Задание к лабораторной работе}

\begin{enumerate}

	\item Реализовать программную процедуру вычисления ИМФ для указанного в цели
		работы класса моделей.

	\item Проверить правильность реализации алгоритмов и работоспособность
		программ.

\end{enumerate}

\section{Теоретический материал}

\subsection{Описание модельной структуры}

Рассмотрим модель стохастической динамической линейной \newline
непрерывно-дискретной системы в пространстве состояний в виде \cite{mono}:
\begin{equation}
  \label{eq:initmod}
  \left\{ 
    \begin{array}{lll}
			\pd{\vec{x}(t)}{t} &= F \vec{x}(t) + C \vec{u}(t) + G \vec{w}(t),
				& t \in [t_0, T], \\
      \vec{y}(t_{k+1}) &= H \vec{x}(t_{k+1}) + \vec{v}(t_{k+1}), 
				& k = 0,\ldots, N-1
    \end{array} 
  \right. 
\end{equation}

Здесь:
\begin{description}
  \item [$\vec{x}(t)$] -- $n$-вектор состояния;
  \item [$F$] -- матрица перехода состояния;
  \item [$\vec{u}(t)$] -- $r$-вектор управления (входного воздействия);
  \item [$C$] -- матрица управления;
  \item [$\vec{w}(t)$] -- $p$-вектор возмущений;
  \item [$G$] -- матрица влияния возмущений;
  \item [$H$] -- матрица наблюдения;
  \item [$\vec{v}(t_{k+1})$] -- $m$-вектор шума измерений;
  \item [$\vec{y}(t_{k+1})$] -- $m$-вектор наблюдений (измерений) отклика;
\end{description}

$F, C, G, H$ --- матрицы соответствующих размеров.

\bigskip
Априорные предположения:
\begin{itemize}
\item $F$ устойчива;
\item пары $(F, C)$ и $(F, G)$ управляемы;
\item пара $(F, H)$ --- наблюдаема;
\item случайные процессы $\{\vec{w}(t), t \in [t_0, T]\}$ и
	$\{\vec{v}(t_{k}), k = 1, 2, \ldots, N\}$ являются стационарными белыми
		гауссовскими шумами, причем:
\[
	E[\vec{w}(t)] = 0,\ E[\vec{w}(t)\vec{w}^{T}(\tau)] = Q \delta(t-\tau)\,
\]
\[
	E[\vec{v}(t_k)] = 0,\ E[\vec{v}(t_k)\vec{v}^{T}(t_j)] = R \delta_{k,j}\,
\]
\[
	E[\vec{v}(t_k)\vec{w}^{T}(t)] = 0,\qquad \forall t_k, t,
\]
		где $\delta(t-\tau)$ --- дельта-функция Дирака, $\delta_{jk}$ --- символ
		Кронекера;

\item начальное состояние $\vec{x}(t_0)$ имеет нормальное распределение с
параметрами $\overline{\vec{x}}(t_0)$ и $P(t_0)$ и не коррелирует с
		$\vec{w(t)}$ и $\vec{v_k}$ при любых значениях $t$ и $k$.

Будем считать, что подлежащие оцениванию параметры $\theta = (\theta_1,
\theta_2, \ldots, \theta_s)$ могут входить в элементы матриц $F, C, G, H, Q, R,
P(t_0)$ и в вектор $\overline{\vec{x}}(t_0)$ в различных комбинациях.

\end{itemize}

\subsection{Информационная матрица Фишера (ИМФ)}

\newcommand{\pred}[0]{t_{k}|t_{k-1}}
\newcommand{\est}[0]{t_k|t_k}
\newcommand{\fut}[0]{t_{k+1}}
\newcommand{\upd}[0]{t_{k+1}|t_{k+1}}
\newcommand{\ol}[1]{\overline{#1}}
\newcommand{\Th}[0]{\Theta}

\newcommand{\sumlim}[2]{\sum\limits_{#1}^{#2}}

Для математической модели (\ref{eq:initmod}) и соответствующими априорными
предположениями, описанными выше, элементы информационной матрицы одноточечного
плана определяются выражением
% TODO: 
\begin{multline}
	\label{eq:fim}
	\hspace{0.5\textwidth} M_{ij}(U;\Th) =\\
		= \sumlim{k=1}{N} \left\{ \Sp \left[ 
		S_0 \left(\Sigma_A(\pred) + \ol{x}_A(\pred) \ol{x}^T_A(\pred) \right)
		S_0^T \pd{H^T}{\theta_j} \inv{B}(t_k) \pd{H}{\theta_i} \right] + \right. \\
		\left.
		+ \Sp\left[ S_0 \left( \Sigma_A(\pred) + \ol{x}(\pred)
		\ol{x}^T(\pred) \right) S_j^T H^T \inv{B}(t_k) \pd{H}{\theta_i}
		\right] + \right. \\ \left.
		+ \Sp\left[ S_i \left( \Sigma_A(\pred) + \ol{x}_A(\pred) \ol{x}^T_A(\pred)
		\right) S_0^T \pd{H^T}{\theta_j} \inv{B}(t_k) H \right] + \right. \\ \left.
		+ \Sp \left[ S_i \left( \Sigma_A(\pred) + \ol{x}_A(\pred) \ol{x}^T_A(\pred)
		\right) S_j^T H^T \inv{B}(t_k) H \right] + \right. \\ \left.
		+ \Sp \left[ \pd{B(t_k)}{\theta_i} \inv{B}(t_k) \pd{B(t_k)}{\theta_j} 
		\right]  \right\},\qquad i,j = 1,2,\ldots,s.
\end{multline}

\subsubsection{Алгоритм вычисления ИМФ}
\begin{enumerate}
	\item Положим $k = 1,\ M(\Theta) = 0$. \\
		Для заданного $\Theta = (\theta_1, \theta_2, \ldots, \theta_s)$ вычислим:
\begin{align*}
	F,\ \pd{F}{\theta_i},\ i=1,2,\ldots,s; \qquad &
	Q,\ \pd{Q}{\theta_i},\ i=1,2,\ldots,s; \\
	C,\ \pd{C}{\theta_i},\ i=1,2,\ldots,s; \qquad &
	R,\ \pd{R}{\theta_i},\ i=1,2,\ldots,s; \\
	G,\ \pd{G}{\theta_i},\ i=1,2,\ldots,s; \qquad &
	\ol{x}_0,\ \pd{\ol{x}_0}{\theta_i},\ i=1,2,\ldots,s; \\
	H,\ \pd{H}{\theta_i},\ i=1,2,\ldots,s; \qquad &
	P_0,\ \pd{P_0}{\theta_i},\ i=1,2,\ldots,s.
\end{align*}

\item Вычислим начальные условия для уравнений чувствительности по следующей
	формуле:
\begin{itemize}
		\item для $j=1$
\begin{equation}
	\label{eq:init_cond_1}
	\begin{bmatrix} 
		P(t_0|t_0) \\ 
		\pd{P(t_0|t_0)}{\theta_1} \\
		\pd{P(t_0|t_0)}{\theta_2} \\
		\vdots \\
		\pd{P(t_0|t_0)}{\theta_s}
	\end{bmatrix} =
	\begin{bmatrix}
		P_0 \\
		\pd{P_0}{\theta_1} \\
		\pd{P_0}{\theta_2} \\
		\vdots \\
		\pd{P_0}{\theta_s}
	\end{bmatrix}
\end{equation}
\item для $j=2,3,\ldots,N$
	\begin{multline}
		\label{eq:init_cond_2}
		\hspace{0.3\textwidth}
		\begin{bmatrix}
			P(t_j|t_j) \\
			\pd{P(t_j|t_j)}{\theta_1} \\
			\vdots \\
			\pd{P(t_j|t_j)}{\theta_s}
		\end{bmatrix} = \\ =
	  \begin{bmatrix}
			\left\{ I - K(t_j) H \right\} P(t_j|t_{j-1}) \\ 
			\left[
				-\pd{K(t_j)}{\theta_1} H - K(t_j) \pd{H}{\theta_1} \right]
			P(t_j|t_{j-1}) + \left\{ I - K(t_j) H \right\}
			\pd{P(t_j|t_{j-1})}{\theta_1} \\
			\vdots \\
			\left[
				-\pd{K(t_j)}{\theta_s} H - K(t_j) \pd{H}{\theta_s} \right]
			P(t_j|t_{j-1}) + \left\{ I - K(t_j) H \right\}
			\pd{P(t_j|t_{j-1})}{\theta_s} 
		\end{bmatrix}.
	\end{multline}
\end{itemize}

\item Решим уравнения чувствительности для ковариационной матрицы оценки
	вектора состояний
\begin{multline*}
	\frac{d}{dt}	
	\begin{bmatrix}
		\hat{x}(t|t_{j-1}) \\
		\pd{\hat{x}(t|t_{j-1})}{\theta_1} \\
		\vdots \\
		\pd{\hat{x}(t|t_{j-1})}{\theta_s} \\
	\end{bmatrix} =
	\begin{bmatrix}
		F \hat{x}(t|t_{j-1}) + C u(t) \\
		\pd{F}{\theta_1} \hat{x}(t|t_{j-1}) + F \pd{\hat{x}(t|t_{j-1})}{\theta_1} +
			\pd{C}{\theta_1} u(t) \\
		\vdots \\
		\pd{F}{\theta_s} \hat{x}(t|t_{j-1}) + F \pd{\hat{x}(t|t_{j-1})}{\theta_s} +
			\pd{C}{\theta_s} u(t) 
	\end{bmatrix} = \\ =
	\begin{bmatrix}
		F & 0 & \cdots & 0 \\
		\pd{F}{\theta_1} & F & \cdots & 0 \\
		\vdots & \vdots & \ddots & \vdots \\
		\pd{F}{\theta_s} & 0 & \cdots & F
	\end{bmatrix}
	\begin{bmatrix}
		\hat{x}(t|t_{j-1}) \\
		\pd{\hat{x}(t|t_{j-1})}{\theta_1} \\
		\vdots \\
		\pd{\hat{x}(t|t_{j-1})}{\theta_s}
	\end{bmatrix} +
	\begin{bmatrix}
		C u(t) \\
		\pd{C}{\theta_1} u(t) \\
		\vdots \\
		\pd{C}{\theta_s} u(t)
	\end{bmatrix},\\ t_{j-1} \le t \le t_j,\ j = 1,\ldots,N.
	\hspace{0.25\textwidth}
\end{multline*}
Получаем $P(t_k|t_{k-1})$ при фиксированном значении вектора неизвестных
параметров $\Theta$.

\item Вычислим
\[
	B(t_k) = H P(t_k|t_{k-1}) H^T + R; \\
\]
\begin{multline*}
	\pd{B(t_j)}{\theta_i} = \pd{H}{\theta_i} P(t_j|t_{j-1}) H^T + H
	\pd{P(t_j|t_{j-1})}{\theta_i} H^T + \\ + H P(t_j|t_{j-1}) \pd{H^T}{\theta_i} +
	\pd{R}{\theta_i},\ i = 1, 2, \ldots, s.
\end{multline*}

\item Вычислим
\[
	K(t_k) = P(t_k|t_{k-1}) H^T \inv{B}(t_k);
\]
\begin{multline*}
	\pd{K(t_k)}{\theta_i} = \pd{P(t_k|t_{k-1})}{\theta_i} H^T \inv{B}(t_k) +
		P(t_k|t_{k-1}) \pd{H^T}{\theta_i} \inv{B}(t_k) + \\ - P(t_k|t_{k-1}) H^T
		\inv{B}(t_k) \pd{B(t_k)}{\theta_i} \inv{B}(t_k),\ i = 1, 2, \ldots, s.
\end{multline*}

\item Вычислим
\[
	\tilde{K}(t_k) = \exp \left[ F(t_{k+1} - t_k) \right] K(t_k) = \Phi_{k+1}
		K(t_k);
\]
\[
	\pd{\tilde{K}(t_k)}{\theta_i} = (t_{k+1} - t_k) \pd{F}{\theta_i} \Phi_{k+1}
		K(t_k) + \Phi_{k+1} \pd{K(t_k)}{\theta_i},\ i = 1, 2, \ldots, s.
\]

\item Вычислим $\Sigma_A(t_k|t_{k-1})$ по следующим формулам:
	\begin{itemize}
		\item если $k = 1$,
			\[
				\Sigma_A(t_1|t_0) = 0,
			\]
		\item если $k > 1$,
			\begin{multline*}
				\Sigma_A(t_k|t_{k-1}) = F_A(t_k, t_{k-1}) \Sigma_A(t_{k-1}|t_{k-2})
					F_A^T(t_k, t_{k-1}) + \\ + \tilde{K}_A(t_{k-1}) B(t_{k-1})
					\tilde{K}_A^T(t_{k-1}).
			\end{multline*}
	\end{itemize}

\item Определим $\ol{x}_A(t_k|t_{k-1})$ при помощи выражения:
	\begin{itemize}
		\item для $k = 1$
			\begin{multline*}
				\ol{x}(t_k|t_{k-1}) =
				\begin{bmatrix}
					\exp[ F(t_1 - t_0) ] \\
					(t_1 - t_0) \pd{F}{\theta_1} \exp[ F(t_1 - t_0) ] \\
					\vdots \\
					(t_1 - t_0) \pd{F}{\theta_s} \exp[ F(t_1 - t_0) ] \\
				\end{bmatrix}
				\ol{x}_0 +
					\begin{bmatrix}
						0 \\
						\exp [ F(t_1 - t_0) ] \pd{\ol{x}_0}{\theta_1} \\
						\vdots \\
						\exp [ F(t_1 - t_0) ] \pd{\ol{x}_0}{\theta_s} \\
					\end{bmatrix}
					\pd{\ol{x}_0}{\theta_i} + \\ +
					\int\limits_{t_0}^{t_1} C_A(t_1, \tau) u(\tau) d\tau,
				\end{multline*}
		\item для $k > 1$
			\[
				F_A (t_k, t_{k-1}) \ol{x}_A(t_{k-1}|t_{k-2}) + 
					\int\limits_{t_{k-1}}^{t_k} C_A(t_k, \tau) u(\tau) d\tau.
			\]
	\end{itemize}

\item Сформируем матрицу $F_A(t_{k+1}, t_k)$ следующим образом:
\begin{gather*}
	F_A(t_k, t_{k-1}) = \\
	\begin{bmatrix}
		\Phi_k & 0 & \cdots & 0 \\
		(t_k - t_{k-1}) \pd{F}{\theta_1} \Phi_k - \tilde{K}(t_{k-1})
			\pd{H}{\theta_1} & \Phi_k - \tilde{K}(t_{k-1}) H & \cdots & 0 \\
		\vdots & \vdots & \ddots & \vdots \\
		(t_k - t_{k-1}) \pd{F}{\theta_s} \Phi_k - \tilde{K}(t_{k-1})
			\pd{H}{\theta_1} & 0 & \cdots & \Phi_k - \tilde{K}(t_{k-1}) H
	\end{bmatrix},
\end{gather*}
сформируем матрицу $C_A(t_{k+1}, \tau)$ следующим образом:
	\[
		C_A(t_k, \tau) =
		\begin{bmatrix}
			\exp[ F(t_k - \tau) ] C \\
			(t_k - \tau) \exp[F(t_k - \tau)] C + \exp[ F(t_k - \tau) ]
				\pd{C}{\theta_1} \\
			\vdots \\
			(t_k - \tau) \exp[F(t_k - \tau)] C + \exp[ F(t_k - \tau) ]
				\pd{C}{\theta_s} 
		\end{bmatrix},
	\]
матрицу $\tilde{K}_A(t_k)$ следующим образом:
\[
	\tilde{K}_A(t_k) =
	\begin{bmatrix}
		\tilde{K}(t_{k-1}) \\
		\pd{\tilde{K}(t_{k-1})}{\theta_1} \\
		\vdots \\
		\pd{\tilde{K}(t_{k-1})}{\theta_s} \\
	\end{bmatrix}.
\]

\item Используя выражение (\ref{eq:fim}), получить приращение 
	$\Delta M(\Theta)$, отвечающее текущему значению $k$.

\item Положим $M(\Theta) = M(\Theta) + \Delta M(\Theta)$.

\item Если $k = N$, то вычисления прекращаются, иначе --- вычислим следующие
	начальные условия по формулам (\ref{eq:init_cond_1})--(\ref{eq:init_cond_2}),
	$k$ заменим на $k + 1$ и осуществим переход на шаг 3.

\end{enumerate}

% imports
\begin{pythontexcustomcode}{py}
import numpy as np
import dill as pkl
from model.model import Model
import pylatex
\end{pythontexcustomcode}

\section{Пример}

% TODO: print numpy arrays here instead of hardcoded matrices
\[
  \begin{pmatrix} x_1(\fut) \\ x_2(\fut) \end{pmatrix} =
  \begin{pmatrix} \theta_1 & 1 \\ 0 & 0.5 \end{pmatrix}
  \begin{pmatrix} x_1(t_k) \\ x_2(t_k) \end{pmatrix}
  + 
	\begin{pmatrix} \theta_2 \\ 1 \end{pmatrix}
	u(t_k) 
  + \begin{pmatrix} w_1(t_k) \\ w_2(t_k) \end{pmatrix}
\]
\[
  y(\fut) =
	\begin{pmatrix} 1 & 0 \end{pmatrix}
	\begin{pmatrix} x_1(\fut) \\ x_2(\fut) \end{pmatrix} +
  \begin{pmatrix} v_1(\fut) \\ v_2(\fut) \end{pmatrix}
\]
\[
  \overline{x}(0) = \begin{pmatrix} 0 \\ 0 \end{pmatrix},\
  P(0) = \begin{pmatrix} 0.1 & 0 \\ 0 & 0.1 \end{pmatrix},\
  Q = \begin{pmatrix} 0.1 & 0 \\ 0 & 0.1 \end{pmatrix},\
  R = 0.1 
\]

Истинные значения параметров
$\theta_{true} = \begin{pmatrix} 0.5 & 0.5 \end{pmatrix}$ \\

\newpage
\subsubsection{Оценивание параметров по произвольному начальному плану}

Спектр непрерывного начального плана:

\begin{pycode}[][fontsize=\small]
filename = './data/plan0.pkl'
with open(filename, 'rb') as f:
    plan0 = pkl.load(f)

def print_plan_tex(plan0):
	print('$')
	print('U = \\left\{')

	for i in range(plan0[0].shape[0]-1):
		x = plan0[0][i]
		xx = pylatex.Matrix(np.round(x.T, 2), mtype='b').dumps()
		print(xx)
		print(',\ ')

	xx = pylatex.Matrix(np.round(plan0[0][-1].T), mtype='b').dumps()
	print(xx)
			
	print('\\right\}')
	print('$')

print_plan_tex(plan0)
\end{pycode}

Веса непрерывного начального плана:
\begin{pycode}
p = np.array(plan0[1], ndmin=2)
p = pylatex.Matrix(p, mtype='b').dumps()
print('$ p = ')
print(p)
print('$')
\end{pycode}

Длина точки плана $N = 20$.

Область планирования: $u(t_k) \in [ -1,\ 1 ],\ k = 0, 1, \ldots, N$.

Область оценивания: $ 0.25 \le \theta_i \le 0.75,\ i = 1, 2 $.

Общее число запусков $\nu = 10$. \\

Веса соответствующего дискретного начального плана:
\begin{pycode}
def round_w(p, v):
	import operator
	p = np.array(p)
	q = len(p)
	sigmaI = np.ceil((v - q) * p)  # 1
	sigmaII = np.floor(v * p)
	vI = v - np.sum(sigmaI)  # 2
	vII = v - np.sum(sigmaII)
	if vI < vII:
			sigma = sigmaI
			v1 = int(vI)
	else:
			sigma = sigmaII
			v1 = int(vII)

	s = np.zeros(q)

	vps = v * p - sigma
	vps_id = [i for i in range(len(vps))]
	vps_t = [(val, key) for val, key in zip(vps, vps_id)]

	vps_t = sorted(vps_t, key=operator.itemgetter(0))
	sorted_id = [elem[1] for elem in vps_t]

	for j in range(q):
			if vps_id[j] in sorted_id[:v1]:
					s[j] = 1
			else:
					s[j] = 0

	p = (sigma + s) / v
	return p

p = round_w(plan0[1], 10)
p = np.array(p, ndmin=2)
p = pylatex.Matrix(p, mtype='b').dumps()
print('$ p = ')
print(p)
print('$')
\end{pycode}

Полученные оценки параметров:
$\hat{\theta} = \begin{bmatrix} 0.342 & 0.559 \end{bmatrix}$.

\newcommand{\rtol}[2]{\frac{||\hat{#1}#2 -
{\hat{#1}#2_{true}}||}{||\hat{#1}#2_{true}||}}

Относительная погрешность в пространстве параметров:
\[
	\delta_{\theta} = \rtol{\theta}{} = 23.9\%.
\]

Средняя относительная погрешность в пространстве откликов:
\[
	\overline{\delta}_Y = \frac{1}{\nu}
		\sum\limits_{i=1}^{q}\sum\limits_{j=1}^{k_i} \rtol{Y}{^{i,j}} = 11.8 \%.
\]

\subsubsection{Синтез оптимального плана и оценивание параметров по плану}

По проведению двойственной процедуры планирования входных сигналов был получен
некоторый оптимальный план. В результате проведения идентификационных экспериментов по
данному плану были получены оценки параметров.

\newpage
Спектр оптимального плана:

\begin{pycode}
filename = './data/plan1.pkl'
with open(filename, 'rb') as f:
    plan1 = pkl.load(f)

print_plan_tex(plan1)
\end{pycode}

То есть в плане одна точка, соответственно, все запуски $\nu$ будут приходиться
на неё.

Полученные оценки параметров:
$\hat{\theta} = \begin{bmatrix} 0.46 & 0.47 \end{bmatrix}$.

Относительная погрешность в пространстве параметров:
\[
	\delta_{\theta} = \rtol{\theta}{} = 6.7\%.
\]

Средняя относительная погрешность в пространстве откликов:
\[
	\overline{\delta}_Y = \frac{1}{\nu}
		\sum\limits_{i=1}^{q}\sum\limits_{j=1}^{k_i} \rtol{Y}{^{i,j}} = 2.6 \%.
\]

\section{Выводы}

По результатам эксперимента видно уменьшение погрешности оценок при оценивании
по оптимальному плану. Таким образом подтверждается эффективность процедуры
активной параметрической идентификации.

\begin{thebibliography}{9}

\begin{sloppypar}

\bibitem{mono} Активная параметрическая идентификация стохастических линейных
  систем: монография / В.И. Денисов, В.М. Чубич, О.С. Черникова, Д.И. Бобылева.
    --- Новосибирск : Изд-во НГТУ, 2009. --- 192 с.
    (Серия <<Монографии НГТУ>>).

\end{sloppypar}

\end{thebibliography}

\end{document}

# vim: ts=2 sw=2
