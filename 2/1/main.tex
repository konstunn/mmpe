
\documentclass[a4paper,14pt]{extarticle}

\usepackage{cmap}

\usepackage[T2A]{fontenc}
\usepackage[utf8x]{inputenc}
\usepackage[russian]{babel}

\usepackage[a4paper,margin=1.5cm,footskip=1cm,left=2cm,right=1.5cm,top=1.5cm
  ,bottom=2.0cm]{geometry}
\usepackage{textcase}
\usepackage{csquotes}
\usepackage{enumitem}

\usepackage[labelsep=period,justification=centering]{caption}

\usepackage{graphicx}
\graphicspath{ {figure/} }

\usepackage{amsmath}
\usepackage{pgfplots}

\usepackage{float}

\usepackage{indentfirst}

\usepackage{textgreek}

\usepackage{pythontex}

\usepackage{comment}

% 
\setlist[description]{leftmargin=\parindent,labelindent=\parindent}

\renewcommand{\baselinestretch}{1.5}

\usepackage[titletoc,title]{appendix}

\DeclareMathOperator{\Sp}{Sp}

\newcommand{\pred}[0]{t_{k+1}|t_k}
\newcommand{\est}[0]{t_k|t_k}
\newcommand{\fut}[0]{t_{k+1}}
\newcommand{\estfut}[0]{t_{k+1}|t_{k+1}}

\renewcommand{\vec}[1]{#1}

\newcommand{\pd}[2]{\frac{\partial #1}{\partial #2}}
\newcommand{\pdpk}[1]{\pd{#1}{\theta_i}}

\newcommand{\inv}[1]{#1^{-1}}

\newcommand{\eps}{\varepsilon}

\begin{document}

\setcounter{secnumdepth}{0}

\begin{titlepage}

  \begin{center}
    Новосибирский государственный технический университет
    
    Факультет прикладной математики и информатики
    
    Кафедра теоретической и прикладной информатики
    
    \vspace{250pt}
    
    \textbf{\LARGE{Лабораторная работа № 1}}
    \medbreak
    \small{по дисциплине \\
    \medbreak
    <<Математические методы планирования эксперимента>>}
    \vspace{100pt}
  \end{center}

  \begin{flushleft}
    \begin{tabbing}
      Группа:\qquad\qquad \= ПММ-61\\
      Студент:            \> Горбунов К. К.\\
      Преподаватель:      \> Чубич В. М.\\
    \end{tabbing}
  \end{flushleft}

  \begin{center}
    \vspace{\fill}
    Новосибирск, 2017 г.
  \end{center}

\end{titlepage}

\newpage

\section{Цель работы}

Ознакомиться с алгоритмами оценивания неизвестных параметров моделей
стохастических линейных дискретных систем.

\section{Порядок выполнения лабораторной работы}

\begin{enumerate}
\item Изучить соответствующий материал лекции на тему <<Оценивание
  неизвестных параметров дискретных моделей>>

\item Последовательно выполнить все задания к лабораторной работе

\item Проверить правильность реализации алгоритмов и работоспособность
  программ не менее чем на пяти тестах

\item Оформить отчет по лабораторной работе
\end{enumerate}

\section{Задание к лабораторной работе}

\begin{enumerate}

\item Реализовать вычисление значения критерия максимального
  \mbox{правдоподобия}

\item Проверить правильность исполнения программы на примерах

\item Для стохастических линейных дискретных моделей при различных
  вариантах вхождения неизвестных параметров в уравнения состояния и измерения
  получить оценки параметров

\end{enumerate}

\newpage

\section{Теоретический материал}

\subsection{Описание модельной структуры}


Модель стохастической динамической линейной дискретной системы в простанстве
состояний в виде \cite{mono}:
\begin{equation}
  \label{eq:initmod}
  \left\{ 
    \begin{array}{lll}
      \vec{x}(t_{k+1}) &= F \vec{x}(t_k) + C \vec{u}(t_k) + G \vec{w}(t_k),&\\
      \vec{y}(t_{k+1}) &= H \vec{x}(t_{k+1}) + \vec{v}(t_{k+1}), 
      & k = 0,\ldots, N-1
    \end{array} 
  \right. 
\end{equation}

Здесь:
\begin{description}
  \item [$\vec{x}(t_k)$] -- $n$-вектор состояния;
  \item [$F$] -- матрица перехода состояния;
  \item [$\vec{u}(t_k)$] -- $r$-вектор управления (входного воздействия);
  \item [$C$] -- матрица управления;
  \item [$\vec{w}(t_k)$] -- $p$-вектор возмущений;
  \item [$G$] -- матрица влияния возмущений;
  \item [$H$] -- матрица наблюдения;
  \item [$\vec{v}(t_{k+1})$] -- $m$-вектор шума измерений;
  \item [$\vec{y}(t_{k+1})$] -- $m$-вектор наблюдений (измерений) отклика;
\end{description}


$F, C, G, H$ --- матрицы соответствующих размеров.

\bigskip
Априорные предположения:
\begin{itemize}
\item $F$ устойчива;
\item пары $(F, C)$ и $(F, G)$ управляемы;
\item пара $(F, H)$ --- наблюдаема;
\item $\vec{w}(t_k)$ и $\vec{v}(t_{k+1})$ --- случайные векторы, образующие
стационарные белые гауссовские последовательности, причем:
\[
E[\vec{w}(t_k)] = 0,\ E[\vec{w}(t_k)\vec{w}^{T}(t_l)] = Q \delta_{k,l}\ ;
\]
\[
E[\vec{v}(t_{k+0}) = 0,\ E[\vec{v}(t_{k+1})\vec{v}^{T}(t_{l+1})] = R
\delta_{k,l}\;
\]
\[
E[\vec{v}(t_k)\vec{w}^{T}(t_k)] = 0,
\]
для любых $k, l = 0, 1, \ldots, N-1$ ($\delta_{k,l}$ --- символ Кронекера);

\item начальное состояние $\vec{x}(0)$ имеет нормальное распределение с
параметрами $\overline{\vec{x}}(0)$ и $P(0)$ и не коррелирует с $\vec{w(t_k)}$
и $\vec{v_{k+1}}$ при любых значениях $k$.

Будем считать, что подлежащие оцениванию параметры $\theta = (\theta_1,
\theta_2, \ldots, \theta_s)$ могут входить в элементы матриц $F, C, G, H, Q, R,
P(0)$ и в вектор $\overline{\vec{x}}(0)$ в различных комбинациях.

\end{itemize}

\subsection{Критерий идентификации}


В качестве критерия идентификации используется логарифмическая функция
правдоподобия. Она имеет вид \cite{mono}:
\begin{equation*}
\begin{split}
  \chi(\theta) = -\ln{L(\theta)} = \frac{Nm}{2}\ ln{2\pi} + \frac{1}{2}
  \sum\limits_{k=0}^{N-1} \left[ \eps^T(t_{k+1}) B^{-1}(t_{k+1}) \eps(t_{k+1}) \right]
  + \\ + \frac{1}{2} \sum\limits_{k=0}^{N-1} \ln \det B^{-1}(t_{k+1}).
\end{split}
\end{equation*}

Алгоритм вычисления критерия следующий \cite{mono}:

\begin{enumerate}
\item для заданного $\theta$  найдем $F, C, G, Q, R, \overline{\vec{x}}(0),
P(0)$.
\item положим $k = 0$, $\chi(\theta) = \frac{Nm}{2} \ln{2\pi}$, $\Delta_2 = 0$,
$P(0|0) = P(0)$.
\item Используя выражения фильтра Калмана, вычислим:
\[
  P(t_{k+1}|t_k) = F P(t_k|t_k) F^T + GQG^T;
\]
\[
  B(t_{k+1}) = H P(t_{k+1}|t_k) H^T + R;
\]
\[
  K(t_{k+1}) = P(t_{k+1}|t_k) H^T B^{-1}(t_{k+1});
\]
\[
  P(t_{k+1}|t_{k+1}) = \left[ I - K(t_{k+1}) H \right] P(t_{k+1}|t_k).
\]
\item Вычислим $\Delta_2 = \Delta_2 + \frac{1}{2} \ln \det B(t_{k+1})$.
\item Увеличим $k$ на единицу. Если $k \le N-1$, перейдем на шаг 3. В противном
случае --- на шаг 6.
\item Положим $k = 0$.
\item Используя выражения фильтра Калмана, найдем:
\[
  \hat{x}(t_{k+1}|t_k) = F \hat{x}(t_k|t_k) + C u(t_k);
\]
\[
  \eps(t_{k+1}) = y(t_{k+1}) - H \hat{x} (t_{k+1}|t_k);
  \hat{x}(t_{k+1}|t_{k+1}) = \hat{x}(t_{k+1}|t_k) + K(t_{k+1}) \eps(t_{k+1}).
\]
\item Найдем значение приращения логарифмической функции правдоподобия
$\Delta_1$, соответствующее текущему значению времени:
\[
  \Delta_1 = \frac{1}{2} \left[ \eps(t_{k+1}) \right]^T B^{-1}(t_{k+1}) 
  \left[ \eps(t_{k+1}) \right].
\]
\item Положим $\chi(\theta) = \chi(\theta) + \Delta_1$.
\item Увеличим $k$ на единицу. Если $k \le N-1$, перейдем на шаг 7.
В противном случае --- на шаг 11.
\item Положим $\chi(\theta) = \chi(\theta) + \Delta_2$ и закончим процесс.
\end{enumerate}

\subsection{Градиент критерия идентификации}

Выражение для градиента критерия имеет вид \cite{mono}:
\begin{equation*}
\begin{split}
  \frac{\partial \chi(\theta)}{\partial \theta_i} = \sum\limits_{k=0}^{N-1}
  \left[ \frac{\partial \eps(t_{k+1})}{\partial \theta_i} \right]^T
  B^{-1}(t_{k+1}) \left[ \eps(t_{k+1}) \right] - \\
  - \frac{1}{2}
  \sum\limits_{k=0}^{N-1} \left[ \eps(t_{k+1}) \right]^T B^{-1}(t_{k+1})
  \frac{\partial B(t_{k+1})}{\partial \theta_i} B^{-1}(t_{k+1}) \eps(t_{k+1}) +
  \\ + 
  \frac{1}{2} \sum\limits_{k=0}^{N-1} \Sp \left[ B^{-1}(t_{k+1})
  \frac{\partial B(t_{k+1})}{\partial \theta_i} \right]. 
\end{split}
\end{equation*}


Алгоритм вычисления градиента критерия следующий \cite{mono}:
\begin{enumerate}
\item Для заданного $\theta$ найдем: $F, \pd{F}{\theta_i}, Q, \pd{Q}{\theta_i},
C, \pd{C}{\theta_i}, R, \pd{R}{\theta_i}, G, \pd{G}{\theta_i}, \overline{x}(0),
\pd{\overline{x}(0)}{\theta_i}, H, \pd{H}{\theta_i}$, $P(0),
\pd{P(0)}{\theta_i}$, $i = 1, 2, \ldots, s$. 

\item Положим $k = 0$; $\pd{\chi(\theta)}{\theta_i} = 0$; $P(0|0) = P(0)$;
$\pd{P(0|0)}{\theta_i} = \pd{P(0)}{\theta_i}$, $i = 1, 2, \ldots, s$;
$\Delta_2' = 0$.

\item Используем выражения фильтра Калмана, вычислим:
\[
  P(t_{k+1}|t_k) = F P(t_k|t_k) F^T + GQG^T;
\]
\[
  B(t_{k+1}) = H P(t_{k+1}|t_k) H^T + R;
\]
\[
  K(t_{k+1}) = P(t_{k+1}|t_k) H^T B^{-1}(t_{k+1});
\]
\[
  P(t_{k+1}|t_{k+1}) = \left[ I - K(t_{k+1}) H \right] P(t_{k+1}|t_k)
\]
\item Найдем $\pd{P(t_{k+1}|t_k)}{\theta_i}$, $\pd{B(t_{k+1})}{\theta_i}$,
$\pd{K(t_{k+1})}{\theta_i}$, $\pd{P(t_{k+1}|t_{k+1})}{\theta_i}$ для всех
$i = 1, 2, \ldots, s$ по формулам, вытекающим из уравнений фильтра Калмана:
\begin{equation*}
\begin{split}
  \pd{P(t_{k+1}|t_k)}{\theta_i} = \pd{F}{\theta_i} P(t_k|t_k) F^T + F
  \pd{P(t_k|t_k)}{\theta_i} F^T + F P(t_k|t_k) \pd{F^T}{\theta_i} + \\ +
  \pd{G}{\theta_i} Q G^T + G \pd{Q}{\theta_i} G^T + G Q \pd{G^T}{\theta_i};
\end{split}
\end{equation*}



\[
  \pdpk{B(t_{k+1})} = \pdpk{H} P(t_{k+1}|t_k) H^T + H \pdpk{P(\pred)} H^T +
  H P(\pred) \pdpk{H^T} + \pdpk{R}; 
\]


\begin{equation*}
\begin{split}
  \pdpk{K(\fut)} = \left[ \pdpk{P(\pred)} H^T + P(\pred) \pdpk{H^T} - \right.
  \\ \left. -
  P(\pred) H^T \inv{B}(\fut) \pdpk{B(\fut)} \right] \inv{B}(\fut); \end{split}
\end{equation*}

\begin{equation*}
\begin{split}
  \pdpk{P(\estfut)} = \left[ I - K(\fut) H \right] \pdpk{P(\pred)} - \\ -
  \left[ \pdpk{K(\fut)} H + K(\fut) \pdpk{H} \right] P(\pred).
\end{split}
\end{equation*}

\item Вычислим $\Delta_2' = \Delta_2' + \frac{1}{2} \Sp \left[ \inv{B}(\fut)
\pdpk{B(\fut)} \right].$

\item Увеличим $k$ на единицу. Если $k \le N-1$, перейдем на шаг 3. В противном
случае --- на шаг 7.

\item Положим $k = 0$.

\item Используя выражения фильтра Калмана, найдем:
\[
  \hat{x}(\pred) = F \hat{x}(\est) + C u(t_k);
\]
\[
  \eps(\fut) = y(\fut) - H \hat{x}(\pred);
\]
\[
  \hat{x}(\estfut) = \hat{x}(\pred) + K(\fut) \eps(\fut).
\]

\item Найдем $\pdpk{\hat{x}(\pred)},\ \pdpk{\eps(\fut)},\
\pdpk{\hat{x}(\estfut)}$ для всех $i = 1, 2, \ldots, s$ по формулам, вытакающим
из уравнений фильтра Калмана:
\[
  \pdpk{\hat{x}(\pred)} = \pdpk{F} \hat{x}(\est) + F \pdpk{\hat{x}(\est)} +
  \pdpk{C} u(t_k);
\]
\[
  \pdpk{\eps(\fut)} = -\pdpk{H} \hat{x}(\pred) - H \pdpk{\hat{x}(\pred)};
\]
\[
  \pdpk{\hat{x}(\estfut)} = \pdpk{\hat{x}(\pred)} + \pdpk{K(\fut)} \eps(\fut) +
  K(\fut) \pdpk{\eps(\fut)}.
\]

\item Найдем значение приращения градиента $\Delta_1'$, соответствующее
текущему значению времени:
\begin{equation*}
\begin{split}
  \Delta_1' = \left[ \pdpk{\eps(\fut)} \right]^T \inv{B}(\fut)
  \left[ \eps(\fut) \right] - \\ - \frac{1}{2} \left[ \eps(\fut) \right]^T
  \inv{B}(\fut) \pdpk{B(\fut)} \inv{B}(\fut) \eps(\fut).
\end{split}
\end{equation*}

\item Положим $\pdpk{\chi(\theta)} = \pdpk{\chi(\theta)} + \Delta_1'$,
$i = 1, 2, \ldots, s$.

\item Увеличим $k$ на единицу. Если $k \le N-1$, перейдем на шаг 8.
В противном случае --- на шаг 13.

\item Положим $\pdpk{\chi(\theta)} = \pdpk{\chi(\theta)} + \Delta_2'$,
$i = 1, 2, \ldots, s$ и закончим процесс.

\end{enumerate}

% imports
\begin{pythontexcustomcode}{py}
import numpy as np
import dill as pkl
from model.model import Model
import pylatex

import matplotlib.pyplot as plt
from mpl_toolkits.mplot3d import Axes3D
from matplotlib import cm

def surf_plot_save(X, Y, Z, filepath):
    fig = plt.figure()
    ax = fig.gca(projection='3d')

    surf = ax.plot_surface(X, Y, Z, cmap=cm.coolwarm,
                           linewidth=0, antialiased=False)

    ax.set_xlabel(r'$\theta_1$')
    ax.set_ylabel(r'$\theta_2$')
    ax.set_zlabel(r'$L(\theta_1, \theta_2)$')

    ax.zaxis.get_major_formatter().set_powerlimits((0, 1))

    plt.savefig(filepath, bbox_inches='tight')
\end{pythontexcustomcode}

\section{Примеры решений}

% define, create model 1
\begin{pycode}[model1]
F = lambda th: [[th[0], 0.],
                [0., th[1]]]

C = lambda th: [[1.0, 0.],
                [0., 1.0]]

G = lambda th: [[1.0, 0.],
                [0., 1.0]]

H = lambda th: [[1.0, 0.],
                [0., 1.0]]

x0_m = lambda th: [[0.],
                   [0.]]

x0_c = lambda th: [[0.1, 0.],
                   [0., 0.1]]

w_c = lambda th: [[0.05, 0.],
                  [0., 0.05]]

v_c = lambda th: [[0.15, 0.],
                  [0., 0.15]]

th_true = [0.5, 0.5]

# TODO: check if there are extra components in 'th'
m = Model(F, C, G, H, x0_m, x0_c, w_c, v_c, th_true)
\end{pycode}

% TODO: introduce counter
% TODO: print numpy arrays instead of matrices
\subsection{Пример 1: параметры в матрице перехода $F$}
\[
  \begin{pmatrix} x_1(\fut) \\ x_2(\fut) \end{pmatrix} =
  \begin{pmatrix} \theta_1 & 0 \\ 0 & \theta_2 \end{pmatrix}
  \begin{pmatrix} x_1(t_k) \\ x_2(t_k) \end{pmatrix}
  + \begin{pmatrix} u_1(t_k) \\ u_2(t_k) \end{pmatrix}
  + \begin{pmatrix} w_1(t_k) \\ w_2(t_k) \end{pmatrix}
\]
\[
  \begin{pmatrix} y_1(\fut) \\ y_2(\fut) \end{pmatrix} =
  \begin{pmatrix} x_1(\fut) \\ x_2(\fut) \end{pmatrix} +
  \begin{pmatrix} v_1(\fut) \\ v_2(\fut) \end{pmatrix}
\]
\[
  \overline{x}(0) = \begin{pmatrix} 0 \\ 0 \end{pmatrix},\
  P(0) = \begin{pmatrix} 0.1 & 0 \\ 0 & 0.1 \end{pmatrix},\
  Q = \begin{pmatrix} 0.05 & 0 \\ 0 & 0.05 \end{pmatrix},\
  R = \begin{pmatrix} 0.15 & 0 \\ 0 & 0.15 \end{pmatrix}
\]

Истинные значения параметров
$\theta_{true} = \begin{pmatrix} 0.5 & 0.5 \end{pmatrix}$ \\

\indent Управляющее воздействие --- дискретная ступенчатая функция
\[
  u(t_k) = \begin{pmatrix} 10 \\ 10 \end{pmatrix},\ k = 0, 1, 2, \ldots, N-1
\]

Область оценивания: $0.1 \le \theta_i \le 0.9$, $i = 1, \ldots, s$.

% simulate response, fit model, dump results
\begin{pycode}[model1]
N = 100
u = np.ones([2, N])
u = u * 10

# run simulation
rez = m.sim(u)  # TODO: return dictionary or named tuple
y = rez[1]

yhat_true = m.yhat(u=u, y=y)

th_init = [0.1, 0.9]

loss_true = m.lik(u=u, y=y)
loss_init = m.lik(u=u, y=y, th=th_init)

rez = m.mle_fit(th=th_init, y=y, u=u)

th_e = rez['x']
loss_e = rez['fun']

yhat_e = m.yhat(u=u, y=y, th=th_e)

data = {'loss_init': loss_init,
        'loss_true': loss_true,
        'th_e': th_e,
        'loss_e': loss_e,
        'th_true': th_true,
        'yhat_true': yhat_true,
        'yhat_e': yhat_e}

with open('./data/data1.pkl', 'wb') as f:
  pkl.dump(data, f)
\end{pycode}

% load results data and print their values below
\begin{pycode}
filepath = './data/data1.pkl'
pytex.add_dependencies(filepath)
with open(filepath, 'rb') as f:
    data = pkl.load(f)

loss_true = data['loss_true']
loss_init = data['loss_init']

loss_e = data['loss_e']

th_e = data['th_e']
th_e_m = np.matrix(th_e)
th_e = np.around(th_e_m, 6)
th_e = pylatex.Matrix(th_e, mtype='b')
th_e = th_e.dumps()

th_t = data['th_true']
th_t = np.matrix(th_t)

rtol = np.linalg.norm(th_t - th_e_m) / np.linalg.norm(th_t)
rtol = "%.6f" % (rtol * 100)

yhat_t = data['yhat_true']
yhat_e = data['yhat_e']
rtoly = np.linalg.norm(yhat_t - yhat_e) / np.linalg.norm(yhat_t)
rtoly = "%.6f" % (rtoly * 100)
\end{pycode}

\subsubsection{Оценивание параметров: результаты}

Значение критерия при истинных значениях параметров
\[
L(\theta_{true}) = \pyc{print("%.3f" % loss_true)}.
\]

Начальное приближение по параметрам
\[ % TODO: make reproducible
\theta_{init} = \begin{bmatrix} 0.1 & 0.9 \end{bmatrix}.
\]

Значение критерия при начальном приближении
\[
L(\theta_{init}) = \pyc{print("%.3f" % loss_init)}.
\]

Полученные оценки параметров
\[
\hat{\theta} = \pyc{print(th_e)}
\]

Значение критерия 
\[
L(\hat{\theta}) = \pyc{print("%.3f" % loss_e)}
\]

\newcommand{\rtol}[1]{\frac{||#1_{true} - \hat{#1}||}{||#1_{true}||}}

Относительная погрешность в пространстве параметров:
\[
\rtol{\theta} = \pyc{print(rtol)} \%
\]

Относительная погрешность в пространстве откликов:
\[
\rtol{y} = \pyc{print(rtoly)} \%
\]

График поверхности, отражающей зависимость критерия от параметров представлен
далее.

% calc for plot
\begin{pycode}[model1]
def f(th0, th1, u, y):
    return m.lik(th=[th0, th1], u=u, y=y)

vf = np.vectorize(f, excluded={'u', 'y'})

X = np.linspace(.1, .9, num=25)
Y = X
X, Y = np.meshgrid(X, Y)

Z = vf(X, Y, u=u, y=y)

with open('./data/plot1.pkl', 'wb') as f:
    pkl.dump([X, Y, Z], f)
\end{pycode}

% save figure
\begin{pycode}[plot1]
depfile = './data/plot1.pkl'
pytex.add_dependencies(depfile)
with open(depfile, 'rb') as f:
    X, Y, Z = pkl.load(f)

surf_plot_save(X, Y, Z, 'figure/L1.pdf')
\end{pycode}

% include figure
\begin{pycode}
pytex.add_dependencies('./figure/L1.pdf')
print(r'\begin{figure}[H]')
print(r'\includegraphics[width=\textwidth]{L1.pdf}')
print(r'\caption{Поверхность, отражающая зависимость критерия от параметров}')
print(r'\end{figure}')
\end{pycode}
Выводы: поверхность выпуклая, виден выраженный минимум.

\newpage % TODO: DO NOT REPEAT YOURSELF
\subsection{Пример 2: параметры в матрице управления $C$}

% define, create model 2
\begin{pycode}[model2]
F = lambda th: [[1.0, 0.],
                [0., 1.0]]

C = lambda th: [[th[0], 0.],
                [0., th[1]]]

G = lambda th: [[1.0, 0.],
                [0., 1.0]]

H = lambda th: [[1.0, 0.],
                [0., 1.0]]

x0_m = lambda th: [[0.],
                   [0.]]

x0_c = lambda th: [[0.1, 0.],
                   [0., 0.1]]

w_c = lambda th: [[0.05, 0.],
                  [0., 0.05]]

v_c = lambda th: [[0.15, 0.],
                  [0., 0.15]]

th_true = [1, 1]

m = Model(F, C, G, H, x0_m, x0_c, w_c, v_c, th_true)
\end{pycode}

\[
  \begin{pmatrix} x_1(\fut) \\ x_2(\fut) \end{pmatrix} =
  \begin{pmatrix} x_1(t_k) \\ x_2(t_k) \end{pmatrix} +
  \begin{pmatrix} \theta_1 & 0 \\ 0 & \theta_2 \end{pmatrix}
  \begin{pmatrix} u_1(t_k) \\ u_2(t_k) \end{pmatrix}
  + \begin{pmatrix} w_1(t_k) \\ w_2(t_k) \end{pmatrix}
\]
\[
  \begin{pmatrix} y_1(\fut) \\ y_2(\fut) \end{pmatrix} =
  \begin{pmatrix} x_1(\fut) \\ x_2(\fut) \end{pmatrix} +
  \begin{pmatrix} v_1(\fut) \\ v_2(\fut) \end{pmatrix}
\]

Истинные значения параметров
$\theta_{true} = \begin{pmatrix} 1 & 1 \end{pmatrix}$ \\

Остальные условия остаются такими же, как в примере 1.

\subsubsection{Оценивание параметров: результаты}

% simulate response, fit model, dump results
\begin{pycode}[model2]
N = 100
u = np.ones([2, N])
u = u * 10

# run simulation
rez = m.sim(u)  # TODO: return dictionary or named tuple
y = rez[1]

yhat_true = m.yhat(u=u, y=y)

th_init = [0.1, 0.9]

loss_true = m.lik(u=u, y=y)
loss_init = m.lik(u=u, y=y, th=th_init)

rez = m.mle_fit(th=th_init, y=y, u=u)

th_e = rez['x']
loss_e = rez['fun']

yhat_e = m.yhat(u=u, y=y, th=th_e)

data = {'loss_init': loss_init,
        'loss_true': loss_true,
        'th_e': th_e,
        'loss_e': loss_e,
        'th_true': th_true,
        'yhat_true': yhat_true,
        'yhat_e': yhat_e}

with open('./data/data2.pkl', 'wb') as f:
    pkl.dump(data, f)
\end{pycode}

% load results data and print their values below
\begin{pycode}
filepath = './data/data2.pkl'
pytex.add_dependencies(filepath)
with open(filepath, 'rb') as f:
  data = pkl.load(f)

loss_true = data['loss_true']
loss_init = data['loss_init']

loss_e = data['loss_e']

th_e = data['th_e']
th_e_m = np.matrix(th_e)
th_e = np.around(th_e_m, 6)
th_e = pylatex.Matrix(th_e, mtype='b')
th_e = th_e.dumps()

th_t = data['th_true']
th_t = np.matrix(th_t)

rtol = np.linalg.norm(th_t - th_e_m) / np.linalg.norm(th_t)
rtol = "%.6f" % (rtol * 100)

yhat_t = data['yhat_true']
yhat_e = data['yhat_e']
rtoly = np.linalg.norm(yhat_t - yhat_e) / np.linalg.norm(yhat_t)
rtoly = "%.6f" % (rtoly * 100)
\end{pycode}

Значение критерия при истинных значениях параметров
\[
L(\theta_{true}) = \pyc{print("%.3f" % loss_true)}.
\]

Начальное приближение по параметрам
\[ % TODO: make reproducible
\theta_{init} = \begin{bmatrix} 0.1 & 0.9 \end{bmatrix}.
\]

Значение критерия при начальном приближении
\[
L(\theta_{init}) = \pyc{print("%.3f" % loss_init)}.
\]

Полученные оценки параметров
\[
\hat{\theta} = \pyc{print(th_e)}
\]

Значение критерия 
\[
L(\hat{\theta}) = \pyc{print("%.3f" % loss_e)}
\]

Относительная погрешность в пространстве параметров:
\[
\rtol{\theta} = \pyc{print(rtol)} \%
\]

Относительная погрешность в пространстве откликов:
\[
\rtol{y} = \pyc{print(rtoly)} \%
\]

График поверхности, отражающей зависимость критерия от параметров представлен
далее.

% calc for plot
\begin{pycode}[model2]
def f(th0, th1, u, y):
    return m.lik(th=[th0, th1], u=u, y=y)

vf = np.vectorize(f, excluded={'u', 'y'})

X = np.linspace(0.5, 1.5, num=25)
Y = X
X, Y = np.meshgrid(X, Y)

Z = vf(X, Y, u=u, y=y)

with open('./data/plot2.pkl', 'wb') as f:
    pkl.dump([X, Y, Z], f)
\end{pycode}

% save figure
\begin{pycode}[plot2]
depfile = './data/plot2.pkl'
pytex.add_dependencies(depfile)
with open(depfile, 'rb') as f:
    X, Y, Z = pkl.load(f)

surf_plot_save(X, Y, Z, 'figure/L2.pdf')
\end{pycode}

% include figure
\begin{pycode}
pytex.add_dependencies('./figure/L2.pdf')
print(r'\begin{figure}[H]')
print(r'\includegraphics[width=\textwidth]{L2.pdf}')
print(r'\caption{Поверхность, отражающая зависимость критерия от параметров}')
print(r'\end{figure}')
\end{pycode}

Вывод: поверхность выпуклая, минимум выраженный, но менее, чем в примере 1.
Это видно при сравнении масштабов осей $Z$ графика в примере 1 и графика в данном
примере.

\newpage
\subsection{Пример 3: параметры в матрице наблюдения $H$}

% define, create model 3
\begin{pycode}[model3]
F = lambda th: [[1.0, 0.],
                [0., 1.0]]

C = lambda th: [[1.0, 0.],
                [0., 1.0]]

G = lambda th: [[1.0, 0.],
                [0., 1.0]]

H = lambda th: [[th[0], 0.],
                [0., th[1]]]

x0_m = lambda th: [[0.],
                   [0.]]

x0_c = lambda th: [[0.1, 0.],
                   [0., 0.1]]

w_c = lambda th: [[0.05, 0.],
                  [0., 0.05]]

v_c = lambda th: [[0.15, 0.],
                  [0., 0.15]]

th_true = [1, 1]

m = Model(F, C, G, H, x0_m, x0_c, w_c, v_c, th_true)
\end{pycode}

\[
  \begin{pmatrix} x_1(\fut) \\ x_2(\fut) \end{pmatrix} =
  \begin{pmatrix} x_1(t_k) \\ x_2(t_k) \end{pmatrix} +
  \begin{pmatrix} u_1(t_k) \\ u_2(t_k) \end{pmatrix}
  + \begin{pmatrix} w_1(t_k) \\ w_2(t_k) \end{pmatrix}
\]
\[
  \begin{pmatrix} y_1(\fut) \\ y_2(\fut) \end{pmatrix} =
  \begin{pmatrix} \theta_1 & 0 \\ 0 & \theta_2 \end{pmatrix}
  \begin{pmatrix} x_1(\fut) \\ x_2(\fut) \end{pmatrix} +
  \begin{pmatrix} v_1(\fut) \\ v_2(\fut) \end{pmatrix}
\]

Истинные значения параметров
$\theta_{true} = \begin{pmatrix} 1 & 1 \end{pmatrix}$ \\

Остальные условия остаются такими же, как в примере 1.

% simulate response, fit model, dump results
\begin{pycode}[model3]
N = 100
u = np.ones([2, N])
u = u * 10

# run simulation
rez = m.sim(u)  # TODO: return dictionary or named tuple
y = rez[1]

yhat_true = m.yhat(u=u, y=y)

th_init = [0.1, 0.9]

loss_true = m.lik(u=u, y=y)
loss_init = m.lik(u=u, y=y, th=th_init)

rez = m.mle_fit(th=th_init, y=y, u=u)

th_e = rez['x']
loss_e = rez['fun']

yhat_e = m.yhat(u=u, y=y, th=th_e)

data = {'loss_init': loss_init,
        'loss_true': loss_true,
        'th_e': th_e,
        'loss_e': loss_e,
        'th_true': th_true,
        'yhat_true': yhat_true,
        'yhat_e': yhat_e}

with open('./data/data3.pkl', 'wb') as f:
    pkl.dump(data, f)
\end{pycode}

% load results data and print their values below
\begin{pycode}
filepath = './data/data3.pkl'
pytex.add_dependencies(filepath)
with open(filepath, 'rb') as f:
  data = pkl.load(f)

loss_true = data['loss_true']
loss_init = data['loss_init']

loss_e = data['loss_e']

th_e = data['th_e']
th_e_m = np.matrix(th_e)
th_e = np.around(th_e_m, 6)
th_e = pylatex.Matrix(th_e, mtype='b')
th_e = th_e.dumps()

th_t = data['th_true']
th_t = np.matrix(th_t)

rtol = np.linalg.norm(th_t - th_e_m) / np.linalg.norm(th_t)
rtol = "%.6f" % (rtol * 100)

yhat_t = data['yhat_true']
yhat_e = data['yhat_e']
rtoly = np.linalg.norm(yhat_t - yhat_e) / np.linalg.norm(yhat_t)
rtoly = "%.6f" % (rtoly * 100)
\end{pycode}

Значение критерия при истинных значениях параметров
\[
L(\theta_{true}) = \pyc{print("%.3f" % loss_true)}.
\]

Начальное приближение по параметрам
\[ % TODO: make reproducible
\theta_{init} = \begin{bmatrix} 0.1 & 0.9 \end{bmatrix}.
\]

Значение критерия при начальном приближении
\[
L(\theta_{init}) = \pyc{print("%.3f" % loss_init)}.
\]

Полученные оценки параметров
\[
\hat{\theta} = \pyc{print(th_e)}
\]

Значение критерия 
\[
L(\hat{\theta}) = \pyc{print("%.3f" % loss_e)}
\]

Относительная погрешность в пространстве параметров:
\[
\rtol{\theta} = \pyc{print(rtol)} \%
\]

Относительная погрешность в пространстве откликов:
\[
\rtol{y} = \pyc{print(rtoly)} \%
\]

График поверхности, отражающей зависимость критерия от параметров представлен
далее.

% calc for plot
\begin{pycode}[model3]
def f(th0, th1, u, y):
    return m.lik(th=[th0, th1], u=u, y=y)

vf = np.vectorize(f, excluded={'u', 'y'})

X = np.linspace(0.5, 2.0, num=25)
Y = X
X, Y = np.meshgrid(X, Y)

Z = vf(X, Y, u=u, y=y)

with open('./data/plot3.pkl', 'wb') as f:
    pkl.dump([X, Y, Z], f)
\end{pycode}

% save figure
\begin{pycode}[plot3]
depfile = './data/plot3.pkl'
pytex.add_dependencies(depfile)
with open(depfile, 'rb') as f:
    X, Y, Z = pkl.load(f)

surf_plot_save(X, Y, Z, 'figure/L3.pdf')
\end{pycode}

% include figure
\begin{pycode}
pytex.add_dependencies('./figure/L3.pdf')
print(r'\begin{figure}[H]')
print(r'\includegraphics[width=\textwidth]{L3.pdf}')
print(r'\caption{Поверхность, отражающая зависимость критерия от параметров}')
print(r'\end{figure}')
\end{pycode}

Вывод: поверхность выпуклая, минимум выраженный.

\subsection{Пример 4: параметры в матрице $Q$}

% define, create model 4
\begin{pycode}[model4]
F = lambda th: [[1.0, 0.],
                [0., 1.0]]

C = lambda th: [[1.0, 0.],
                [0., 1.0]]

G = lambda th: [[1.0, 0.],
                [0., 1.0]]

H = lambda th: [[1.0, 0.],
                [0., 1.0]]

x0_m = lambda th: [[0.],
                   [0.]]

x0_c = lambda th: [[0.1, 0.],
                   [0., 0.1]]

w_c = lambda th: [[th[0], 0.],
                  [0., th[1]]]

v_c = lambda th: [[0.15, 0.],
                  [0., 0.15]]

th_true = [0.1, 0.1]

m = Model(F, C, G, H, x0_m, x0_c, w_c, v_c, th_true)
\end{pycode}

\[
  \begin{pmatrix} x_1(\fut) \\ x_2(\fut) \end{pmatrix} =
  \begin{pmatrix} x_1(t_k) \\ x_2(t_k) \end{pmatrix} +
  \begin{pmatrix} u_1(t_k) \\ u_2(t_k) \end{pmatrix}
  + \begin{pmatrix} w_1(t_k) \\ w_2(t_k) \end{pmatrix}
\]
\[
  \begin{pmatrix} y_1(\fut) \\ y_2(\fut) \end{pmatrix} =
  \begin{pmatrix} x_1(\fut) \\ x_2(\fut) \end{pmatrix} +
  \begin{pmatrix} v_1(\fut) \\ v_2(\fut) \end{pmatrix}
\]
\[
  Q = \begin{pmatrix} \theta_1 & 0 \\ 0 & \theta_2 \end{pmatrix},\
\]

Истинные значения параметров
$\theta_{true} = \begin{pmatrix} 0.1 & 0.1 \end{pmatrix}$

Область оценивания: $0.01 \le \theta_i \le 0.2$, $i = 1, \ldots, s$.

Остальные условия остаются такими же, как в примере 1.

% simulate response, fit model, dump results
\begin{pycode}[model4]
N = 100
u = np.ones([2, N])
u = u * 10

# run simulation
rez = m.sim(u)  # TODO: return dictionary or named tuple
y = rez[1]

yhat_true = m.yhat(u=u, y=y)

th_init = [0.2, 0.1]

loss_true = m.lik(u=u, y=y)
loss_init = m.lik(u=u, y=y, th=th_init)

bounds = [(0.01, 0.2)] * 2

rez = m.mle_fit(th=th_init, y=y, u=u, bounds=bounds)

th_e = rez['x']
loss_e = rez['fun']

yhat_e = m.yhat(u=u, y=y, th=th_e)

data = {'loss_init': loss_init,
        'loss_true': loss_true,
        'th_e': th_e,
        'loss_e': loss_e,
        'th_true': th_true,
        'yhat_true': yhat_true,
        'yhat_e': yhat_e}

with open('./data/data4.pkl', 'wb') as f:
    pkl.dump(data, f)
\end{pycode}

% load results data and print their values below
\begin{pycode}
filepath = './data/data4.pkl'
pytex.add_dependencies(filepath)
with open(filepath, 'rb') as f:
  data = pkl.load(f)

loss_true = data['loss_true']
loss_init = data['loss_init']

loss_e = data['loss_e']

th_e = data['th_e']
th_e_m = np.matrix(th_e)
th_e = np.around(th_e_m, 6)
th_e = pylatex.Matrix(th_e, mtype='b')
th_e = th_e.dumps()

th_t = data['th_true']
th_t = np.matrix(th_t)

rtol = np.linalg.norm(th_t - th_e_m) / np.linalg.norm(th_t)
rtol = "%.6f" % (rtol * 100)

yhat_t = data['yhat_true']
yhat_e = data['yhat_e']
rtoly = np.linalg.norm(yhat_t - yhat_e) / np.linalg.norm(yhat_t)
rtoly = "%.6f" % (rtoly * 100)
\end{pycode}

Значение критерия при истинных значениях параметров
\[
L(\theta_{true}) = \pyc{print("%.3f" % loss_true)}.
\]

Начальное приближение по параметрам
\[ % TODO: make reproducible
\theta_{init} = \begin{bmatrix} 0.2 & 0.1 \end{bmatrix}.
\]

Значение критерия при начальном приближении
\[
L(\theta_{init}) = \pyc{print("%.3f" % loss_init)}.
\]

Полученные оценки параметров
\[
\hat{\theta} = \pyc{print(th_e)}
\]

Значение критерия 
\[
L(\hat{\theta}) = \pyc{print("%.3f" % loss_e)}
\]

Относительная погрешность в пространстве параметров:
\[
\rtol{\theta} = \pyc{print(rtol)} \%
\]

Относительная погрешность в пространстве откликов:
\[
\rtol{y} = \pyc{print(rtoly)} \%
\]

График поверхности, отражающей зависимость критерия от параметров представлен
далее.

% calc for plot
\begin{pycode}[model4]
def f(th0, th1, u, y):
    return m.lik(th=[th0, th1], u=u, y=y)

vf = np.vectorize(f, excluded={'u', 'y'})

X = np.linspace(0.01, 0.2, num=25)
Y = X
X, Y = np.meshgrid(X, Y)

Z = vf(X, Y, u=u, y=y)

with open('./data/plot4.pkl', 'wb') as f:
    pkl.dump([X, Y, Z], f)
\end{pycode}

% save figure
\begin{pycode}[plot4]
depfile = './data/plot4.pkl'
pytex.add_dependencies(depfile)
with open(depfile, 'rb') as f:
    X, Y, Z = pkl.load(f)

surf_plot_save(X, Y, Z, 'figure/L4.pdf')
\end{pycode}

% include figure
\begin{pycode}
pytex.add_dependencies('./figure/L4.pdf')
print(r'\begin{figure}[H]')
print(r'\includegraphics[width=\textwidth]{L4.pdf}')
print(r'\caption{Поверхность, отражающая зависимость критерия от параметров}')
print(r'\end{figure}')
\end{pycode}

Поверхность невыпуклая, минимума в окрестности истинных значений
параметров нет.

\subsection{Пример 5: параметры в матрице $R$}


\begin{thebibliography}{9}

\section{Заключение}

Матрицы перехода, управления, наблюдений хорошо поддаются оцениванию.
Ковариационные матрицы шумов объекта, наблюдений плохо поддаются оцениванию.

\begin{sloppypar}

\bibitem{mono} Активная параметрическая идентификация стохастических линейных
  систем: монография / В.И. Денисов, В.М. Чубич, О.С. Черникова, Д.И. Бобылева.
    --- Новосибирск : Изд-во НГТУ, 2009. --- 192 с.
    (Серия <<Монографии НГТУ>>).

\end{sloppypar}

\end{thebibliography}

\renewcommand{\baselinestretch}{1}

\begin{appendices}

\section{Исходные тексты программ}

% TODO: make pycode, within it print pyverbatim, read source file and print it
\begin{pyverbatim}[][fontsize=\small]

import math
import tensorflow as tf
import control
import numpy as np
from tensorflow.contrib.distributions import MultivariateNormalFullCovariance
import scipy


class Model(object):

    # TODO: introduce some more default argument values, check types, cast if
    # neccessary
    def __init__(self, F, C, G, H, x0_mean, x0_cov, w_cov, v_cov, th):
        """
        Arguments are all callables (functions) of 'th' returning python lists
        except for 'th' itself (of course)
        """

        # TODO: evaluate and cast everything to numpy matrices first
        # TODO: cast floats, ints to numpy matrices
        # TODO: allow both constant matrices and callables

        # store arguments, after that check them
        self.__F = F
        self.__C = C
        self.__G = G
        self.__H = H
        self.__x0_mean = x0_mean
        self.__x0_cov = x0_cov
        self.__w_cov = w_cov
        self.__v_cov = v_cov
        self.__th = th

        # evaluate all functions
        th = np.array(th)
        F = np.array(F(th))
        C = np.array(C(th))
        H = np.array(H(th))
        G = np.array(G(th))
        w_cov = np.array(w_cov(th))    # Q
        v_cov = np.array(v_cov(th))    # R
        x0_m = np.array(x0_mean(th))
        x0_cov = np.array(x0_cov(th))  # P_0

        # get dimensions and store them as well
        self.__n = n = F.shape[0]
        self.__m = m = H.shape[0]
        self.__p = p = G.shape[1]
        self.__r = r = C.shape[1]

        # generate means
        w_mean = np.zeros([p, 1], np.float64)
        v_mean = np.zeros([m, 1], np.float64)

        # and store them
        self.__w_mean = w_mean
        self.__v_mean = v_mean

        # check conformability
        u = np.ones([r, 1])
        # generate random vectors
        # squeeze, because mean must be one dimensional
        x = np.random.multivariate_normal(x0_m.squeeze(), x0_cov)
        w = np.random.multivariate_normal(w_mean.squeeze(), w_cov)
        v = np.random.multivariate_normal(v_mean.squeeze(), v_cov)

        # shape them as column-vectors
        x = x.reshape([n, 1])
        w = w.reshape([p, 1])
        v = v.reshape([m, 1])

        # if model is not conformable, exception would be raised (thrown) here
        F * x + C * u + G * w
        H * x + v

        # check controllability, stability, observability
        self.__validate()

        # if the execution reached here, all is fine so
        # define corresponding computational tensorflow graphs
        self.__define_observations_simulation()
        self.__define_likelihood_computation()

    def __define_observations_simulation(self):
        # TODO: reduce code not to create extra operations

        self.__sim_graph = tf.Graph()
        sim_graph = self.__sim_graph

        r = self.__r
        m = self.__m
        n = self.__n
        p = self.__p

        x0_mean = self.__x0_mean
        x0_cov = self.__x0_cov

        with sim_graph.as_default():

            th = tf.placeholder(tf.float64, shape=[None], name='th')

            # TODO: this should be continuous function of time
            # but try to let pass array also
            u = tf.placeholder(tf.float64, shape=[r, None], name='u')

            t = tf.placeholder(tf.float64, shape=[None], name='t')

            # TODO: refactor

            # FIXME: gradient of py_func is None
            # TODO: embed function itself in the graph, must rebuild the graph
            # if the structure of the model change
            # use tf.convert_to_tensor
            F = tf.convert_to_tensor(self.__F(th), tf.float64)
            F.set_shape([n, n])

            C = tf.convert_to_tensor(self.__C(th), tf.float64)
            C.set_shape([n, r])

            G = tf.convert_to_tensor(self.__G(th), tf.float64)
            G.set_shape([n, p])

            H = tf.convert_to_tensor(self.__H(th), tf.float64)
            H.set_shape([m, n])

            x0_mean = tf.convert_to_tensor(x0_mean(th), tf.float64)
            x0_mean = tf.squeeze(x0_mean)

            x0_cov = tf.convert_to_tensor(x0_cov(th), tf.float64)
            x0_cov.set_shape([n, n])

            x0_dist = MultivariateNormalFullCovariance(x0_mean, x0_cov,
                                                       name='x0_dist')

            Q = tf.convert_to_tensor(self.__w_cov(th), tf.float64)
            Q.set_shape([p, p])

            w_mean = self.__w_mean.squeeze()
            w_dist = MultivariateNormalFullCovariance(w_mean, Q, name='w_dist')

            R = tf.convert_to_tensor(self.__v_cov(th), tf.float64)
            R.set_shape([m, m])
            v_mean = self.__v_mean.squeeze()
            v_dist = MultivariateNormalFullCovariance(v_mean, R, name='v_dist')

            def sim_obs(x):
                v = v_dist.sample()
                v = tf.reshape(v, [m, 1])
                y = H @ x + v  # the syntax is valid for Python >= 3.5
                return y

            def sim_loop_cond(x, y, t, k):
                N = tf.stack([tf.shape(t)[0]])
                N = tf.reshape(N, ())
                return tf.less(k, N-1)

            def sim_loop_body(x, y, t, k):

                # TODO: this should be function of time
                u_t_k = tf.slice(u, [0, k], [r, 1])

                def state_propagate(x):
                    w = w_dist.sample()
                    w = tf.reshape(w, [p, 1])
                    Fx = tf.matmul(F, x, name='Fx')
                    Cu = tf.matmul(C, u_t_k, name='Cu')
                    Gw = tf.matmul(G, w, name='Gw')
                    x = Fx + Cu + Gw
                    return x

                tk = tf.slice(t, [k], [2], 'tk')

                x_k = x[:, -1]
                x_k = tf.reshape(x_k, [n, 1])

                x_k = state_propagate(x_k)

                y_k = sim_obs(x_k)

                # TODO: stack instead of concat
                x = tf.concat([x, x_k], 1)
                y = tf.concat([y, y_k], 1)

                k = k + 1

                return x, y, t, k

            x = x0_dist.sample(name='x0_sample')
            x = tf.reshape(x, [n, 1], name='x')

            # this zeroth measurement should be thrown away
            y = sim_obs(x)
            k = tf.constant(0, name='k')

            shape_invariants = [tf.TensorShape([n, None]),
                                tf.TensorShape([m, None]),
                                t.get_shape(),
                                k.get_shape()]

            sim_loop = tf.while_loop(sim_loop_cond, sim_loop_body,
                                     [x, y, t, k], shape_invariants,
                                     name='sim_loop')

            self.__sim_loop_op = sim_loop

    # defines graph
    def __define_likelihood_computation(self):

        self.__lik_graph = tf.Graph()
        lik_graph = self.__lik_graph

        r = self.__r
        m = self.__m
        n = self.__n
        p = self.__p

        x0_mean = self.__x0_mean
        x0_cov = self.__x0_cov

        with lik_graph.as_default():
            # FIXME: Don't Repeat Yourself (in simulation and here)
            th = tf.placeholder(tf.float64, shape=[None], name='th')
            u = tf.placeholder(tf.float64, shape=[r, None], name='u')
            t = tf.placeholder(tf.float64, shape=[None], name='t')
            y = tf.placeholder(tf.float64, shape=[m, None], name='y')

            N = tf.stack([tf.shape(t)[0]])
            N = tf.reshape(N, ())

            F = tf.convert_to_tensor(self.__F(th), tf.float64)
            F.set_shape([n, n])

            C = tf.convert_to_tensor(self.__C(th), tf.float64)
            C.set_shape([n, r])

            G = tf.convert_to_tensor(self.__G(th), tf.float64)
            G.set_shape([n, p])

            H = tf.convert_to_tensor(self.__H(th), tf.float64)
            H.set_shape([m, n])

            x0_mean = tf.convert_to_tensor(x0_mean(th), tf.float64)
            x0_mean.set_shape([n, 1])

            P_0 = tf.convert_to_tensor(x0_cov(th), tf.float64)
            P_0.set_shape([n, n])

            Q = tf.convert_to_tensor(self.__w_cov(th), tf.float64)
            Q.set_shape([p, p])

            R = tf.convert_to_tensor(self.__v_cov(th), tf.float64)
            R.set_shape([m, m])

            I = tf.eye(n, n, dtype=tf.float64)

            def lik_loop_cond(k, P, S, t, u, x, y):
                return tf.less(k, N-1)

            def lik_loop_body(k, P, S, t, u, x, y):

                # TODO: this should be function of time
                u_t_k = tf.slice(u, [0, k], [r, 1])

                # k+1, cause zeroth measurement should not be taken into account
                y_k = tf.slice(y, [0, k+1], [m, 1])

                t_k = tf.slice(t, [k], [2], 't_k')

                # TODO: extract Kalman filter to a separate class
                def state_predict(x):
                    Fx = tf.matmul(F, x, name='Fx')
                    Cu = tf.matmul(C, u_t_k, name='Cu')
                    x = Fx + Cu
                    return x

                def covariance_predict(P):
                    GQtG = tf.matmul(G @ Q, G, transpose_b=True)
                    PtF = tf.matmul(P, F, transpose_b=True)
                    P = tf.matmul(F, P) + PtF + GQtG
                    return P

                x = state_predict(x)

                P = covariance_predict(P)

                E = y_k - tf.matmul(H, x)

                B = tf.matmul(H @ P, H, transpose_b=True) + R
                invB = tf.matrix_inverse(B)

                K = tf.matmul(P, H, transpose_b=True) @ invB

                S_k = tf.matmul(E, invB @ E, transpose_a=True)
                S_k = 0.5 * (S_k + tf.log(tf.matrix_determinant(B)))

                S = S + S_k

                # state update
                x = x + tf.matmul(K, E)

                # covariance update
                P = (I - K @ H) @ P

                k = k + 1

                return k, P, S, t, u, x, y

            k = tf.constant(0, name='k')
            P = P_0
            S = tf.constant(0.0, dtype=tf.float64, shape=[1, 1], name='S')
            x = x0_mean

            # TODO: make a named tuple of named list
            lik_loop = tf.while_loop(lik_loop_cond, lik_loop_body,
                                     [k, P, S, t, u, x, y], name='lik_loop')

            dS = tf.gradients(lik_loop[2], th)

            self.__lik_loop_op = lik_loop
            self.__dS = dS

    def __isObservable(self, th=None):
        if th is None:
            th = self.__th
        F = np.array(self.__F(th))
        C = np.array(self.__C(th))
        n = self.__n
        obsv_matrix = control.obsv(F, C)
        rank = np.linalg.matrix_rank(obsv_matrix)
        return rank == n

    def __isControllable(self, th=None):
        if th is None:
            th = self.__th
        F = np.array(self.__F(th))
        C = np.array(self.__C(th))
        n = self.__n
        ctrb_matrix = control.ctrb(F, C)
        rank = np.linalg.matrix_rank(ctrb_matrix)
        return rank == n

    # FIXME: fix to discrete
    def __isStable(self, th=None):
        if th is None:
            th = self.__th
        F = np.array(self.__F(th))
        eigv = np.linalg.eigvals(F)
        real_parts = np.real(eigv)
        return np.all(real_parts < 0)

    def __validate(self, th=None):
        # FIXME: do not raise exceptions
        # TODO: prove, print matrices and their criteria
        if not self.__isControllable(th):
            # raise Exception('''Model is not controllable. Set different
            #                structure or parameters values''')
            pass

        if not self.__isStable(th):
            # raise Exception('''Model is not stable. Set different structure or
            #                parameters values''')
            pass

        if not self.__isObservable(th):
            # raise Exception('''Model is not observable. Set different
            #                structure or parameters values''')
            pass

    def sim(self, u, th=None):
        if th is None:
            th = self.__th

        k = u.shape[1]
        t = np.linspace(0, k-1, k)

        self.__validate(th)
        g = self.__sim_graph

        if t.shape[0] != u.shape[1]:
            raise Exception('''t.shape[0] != u.shape[1]''')

        # run simulation graph
        with tf.Session(graph=g) as sess:
            t_ph = g.get_tensor_by_name('t:0')
            th_ph = g.get_tensor_by_name('th:0')
            u_ph = g.get_tensor_by_name('u:0')
            rez = sess.run(self.__sim_loop_op, {th_ph: th, t_ph: t, u_ph: u})

        return rez

    def lik(self, u, y, th=None):

        # hack continuous to discrete system
        k = u.shape[1]
        t = np.linspace(0, k-1, k)

        if th is None:
            th = self.__th

        # to numpy 1D array
        th = np.array(th).squeeze()

        # self.__validate(th)
        g = self.__lik_graph

        # TODO: check for y also
        if t.shape[0] != u.shape[1]:
            raise Exception('''t.shape[0] != u.shape[1]''')

        # run lik graph
        with tf.Session(graph=g) as sess:
            t_ph = g.get_tensor_by_name('t:0')
            th_ph = g.get_tensor_by_name('th:0')
            u_ph = g.get_tensor_by_name('u:0')
            y_ph = g.get_tensor_by_name('y:0')
            rez = sess.run(self.__lik_loop_op, {th_ph: th, t_ph: t, u_ph: u,
                                                y_ph: y})

        # FIXME: fix to discrete
        N = len(t)
        m = y.shape[0]
        S = rez[2]
        S = S + N*m * 0.5 + np.log(2*math.pi)

        return S

    def __L(self, th, u, y):
        return self.lik(u, y, th)

    def __dL(self, th, u, y):
        return self.dL(u, y, th)

    def dL(self, u, y, th=None):
        if th is None:
            th = self.__th

        # hack continuous to discrete system
        k = u.shape[1]
        t = np.linspace(0, k-1, k)

        # to 1D numpy array
        th = np.array(th).squeeze()

        # self.__validate(th)
        g = self.__lik_graph

        if t.shape[0] != u.shape[1]:
            raise Exception('''t.shape[0] != u.shape[1]''')

        # run lik graph
        with tf.Session(graph=g) as sess:
            t_ph = g.get_tensor_by_name('t:0')
            th_ph = g.get_tensor_by_name('th:0')
            u_ph = g.get_tensor_by_name('u:0')
            y_ph = g.get_tensor_by_name('y:0')
            rez = sess.run(self.__dS, {th_ph: th, t_ph: t, u_ph: u, y_ph: y})

        return rez[0]

    def mle_fit(self, th, u, y):
        # TODO: call slsqp
        th0 = th
        th = scipy.optimize.minimize(self.__L, th0, args=(u, y),
                                     jac=self.__dL, options={'disp': True})
        return th
\end{pyverbatim}

\end{appendices}

\end{document}

# vim: ts=2 sw=2
